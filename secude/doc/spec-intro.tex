\section{Introduction}
\label{if}
\markboth{Introduction}{Introduction}
\thispagestyle{myheadings}

The security techniques used within SecuDE are essentially based on
\be
\m application of {\em asymmetric} and {\em symmetric encryption} in order to guarantee
   confidentiality and authenticity of user information,
\m {\em personal secure storage} of security relevant information (particularly of keys),
\m application of {\em X.500ff}-defined {\em certification} techniques and -formats in order to
   guarantee authenticity of keys and other security relevant information.
\ee
The following sections 
contain ASN.1 definitions of necessary security related objects and
describe basic functions in terms of C-language procedure calls. 
The latter constitute a basic set of modules
which are intended to support {\em security applications},
in that they provide {\em security application programming interfaces}.
The applications {\em X.400} (Electronic Mail) and {\em PEM} (Privacy Enhanced Mail, RFC111x)
are based on these functions. The modules comprise the interfaces {\em Secure-IF},
{\em AF-IF}, {\em X400-IF}, and {\em PEM-IF} as shown in Fig.~\ref{bild1}. 
This interface structure was chosen for the following reasons: \\ [1ex]
Security applications should be independent of underlying technology, i.e. it should
not make any difference for the security functionality in an application whether 
cryptographic algorithms are realized
in software or hardware chips or in external devices like smartcards or smartcard readers,
nor should it make a difference how and where security relevant information like keys 
are stored. They need an interface which largely hides the implications of particular
technology and which is oriented on user required security characteristics
in order to support migrations to advanced technologies. For this purpose
we defined the interface {\em Secure-IF} which is described in chapter~\ref{sec-if}. 
Secure-IF provides {\em Cryptographic Functions} 
and {\em Personal Secure Storage} (PSE) in a technology independent form. \\ [1ex]
{\em Personal Secure Storage} can be realized by various methods. The highest level of
security against unallowed manipulation of security relevant information requires
hardware support. By current standards, {\em smartcards} are appropriate means to
store security relevant personal data because they combine the two aspects
\bi
\m security against eavesdropping through hardware properties and
\m mobility and portability of personal information.
\ei
However, smartcard technology
is currently developing fast, and standardisation in this area is not stable
yet. For this reason, we provide a software substitute of the PSE (SW-PSE). 
We assume that all security relevant functions are
performed by software (e.g. in work stations, PC's or main frames) and that all
security relevant information is stored on disks or other background memory
of the systems. Protection mechanisms as they are offered by smartcards are to
be modelled by means of file encryption and electronic signatures.
The software PSE is described in chapter~\ref{sw-sc}. \\  [1ex]
Later versions, however, will be characterized by a move of security functionality
into specialized hardware, and also smartcards will be used. In this case it
makes sense to define an interface for a smartcard environment application ({\em SCA-IF}).
This interface is supposed to serve security devices, consisting of smartcard
readers and multifunctional smartcards, which perform all security functions
like encryption and decryption of information, signing messages, and verifying
electronic signatures. The interface {\em SCA-IF}, however, is currently not part of the
current specification of SecuDE. \\ [1ex]
In addition, a key pool is assumed where keys (which means the key itself plus additional
information like algorithm identifier) can be stored and restored 
by means of key references. The basic idea is
to have a protected area (or an external security device) for keys which does not allow
to pass secret keys through the Secure-IF in plaintext to the outside world. 
Instead, keys are handled only via
key references and may be moved through the Secure-IF to the application program
only in encrypted form. The Secure-IF supports this kind of key handling.
\\ [1em]

\begin{figure}
\framebox[2.95cm][c]{\rule[-0.5cm]{0cm}{0.9cm}X.400}
\hspace*{0.4cm}
\framebox[2.9cm][c]{\rule[-0.5cm]{0cm}{0.9cm}PEM}
\hspace*{0.4cm}
\framebox[2.8cm][c]{\rule[-0.5cm]{0cm}{0.9cm}Key Managm.}
\hspace*{0.4cm}
\framebox[2.5cm][c]{\rule[-0.5cm]{0cm}{0.9cm}Other Appl.}
\vs{0.3cm}
{\bf
$|$ -- X400-IF -- $|$ \hspace*{0.2cm} $|$ -- PEM-IF -- $|$ \hspace*{0.25cm} $|$ -- KM-IF -- $|$
}
\vs{0.3cm}
\framebox[2.95cm][c]{\rule[-0.5cm]{0cm}{0.9cm}X.400 Support}
\hspace*{0.4cm}
\framebox[2.9cm][c]{\rule[-0.5cm]{0cm}{0.9cm}PEM Support}
\hspace*{0.4cm}
\framebox[2.8cm][c]{\rule[-0.5cm]{0cm}{0.9cm}KM Support}
\vs{0.3cm}
{\bf
$|$ -- -- -- -- -- -- -- -- -- -- AF-IF -- -- -- -- -- -- -- -- -- -- $|$
}
\vs{0.3cm}
\framebox[5.5cm][c]{\rule[-0.5cm]{0cm}{0.9cm}Local Certificate Handling}
\hspace*{0.4cm}
\framebox[3.65cm][c]{\rule[-0.5cm]{0cm}{0.9cm}Directory Access}
\hspace*{0.4cm}
\framebox[3.4cm][c]{\rule[-0.5cm]{0cm}{0.9cm}Other Functions}
\vs{0.3cm}
{\bf
$|$ -- -- -- -- -- -- -- -- -- -- -- -- -- -- Secure-IF -- -- -- -- -- -- -- -- -- -- -- -- -- -- $|$
}
\vs{0.3cm}
\framebox[13.6cm][c]{\rule[-0.5cm]{0cm}{0.9cm}PSE and Cryptography (technology dependent)}
\vs{0.3cm}
{\bf 
$|$ -- -- SCA-IF -- -- $|$
}
\vs{0.3cm}
\framebox[3.8cm][c]{\rule[-0.5cm]{0cm}{0.9cm}SCA-Environment}
\hspace*{4.25cm}
\framebox[2.5cm][c]{\rule[-0.5cm]{0cm}{0.9cm}Crypto-SW}
\hspace*{0.2cm}
\framebox[2.5cm][c]{\rule[-0.5cm]{0cm}{0.9cm}SW-PSE}
%\vs{0.5cm}
%\caption{Interface Structure}
\label{bild1}
\\[1em]
%\label{fig-isw-sm}
\stepcounter{Abb}
{\footnotesize Fig.\arabic{Abb}:
Software Modules Interface Structure}
\end{figure}

Security of the applications X.400 and PEM is based on certification procedures and formats
as defined in this specification in accordance to {\em CCITT X.500ff} and clarified in
{\em Vol. 1: Principles of Security Operations}. An essential part here 
is the {\em Authentication Framework} X.509. For this functionality we defined the 
Local certificate Handling Module with the interface
{\em AF-IF} (AF stands for Authentication Framework) which is described in chapter~\ref{af-if}. 
It can be expected, however, that
the functionality defined by {\em AF-IF} is not limited to the purpose of Electronic Mail, but that
it is useful for a wide range of applications. On the other hand, one can imagine other
applications, e.g. using ECMA certification methods, which use {\em Secure-IF} directly. \\ [0.5cm]
The interface {\em PEM-IF} was defined for the purpose of the application {\em Privacy Enhanced 
Mail}. It comprises functions which are necessary in the context of RFC 1421 - 1423.
{\em PEM-IF} is described in chapter~\ref{pem-if}. 
\\ [1em]
The interface {\em X400-IF} was defined for the purpose of the application
{\em Electronic Mail} due to the {\em CCITT X.400-1988}
recommendations.
It comprises functions which enable an X.400 programmer
to add X.400 security elements into an existing X.400 implementation.
\\[1em]
Naturally, a security system based on asymmetric cryptography
depends on a well organized {\em key management} (KM) infrastructure
including certification authorities, users and directories.
For this purpose, the KM module is intended, comprising
functions to generate keys and certificates,
and other functions to support the secure exchange of security informations
such as keys and certificates. The KM module is not defined yet.
\\[1em]
Chapter~\ref{aux-if} descibes a collection of useful auxiliary functions mostly destined for the
handling of octetstrings, object identifiers, algorithm identifiers, and PEM-specific
transformations. Some of them are of internal use only, and the application programmer will
not need to handle them, but they are mentioned here for completeness.
