\section{Introduction}
\markboth{Introduction}{Introduction}
\thispagestyle{myheadings}
\label{introduction}
Since computers and computer communication are increasingly being used to store and
exchange assets, measures have to be introduced to protect those assets
against accidental or intentional loss, alteration and misuse. The assets to be protected
can be of very different nature. They may consist of system resources owned by service providers
who offer access to data bases, information systems or general computer services, they may comprise
information which is stored and exchanged with the help of computers, and they may even comprise 
electronically signed and stored contracts by which other parties can be held responsible, or
other electronically processed information which guarantees bindings of responsibilities
or legal bindings in organizational contexts. 

The latter examples show that new security techniques are not only needed in areas
where classical security means as the usage of passwords have been proven to be 
unsufficient. Security is needed for areas where the application of computer
communication is going to: as a means for cooperative work which may
span over organizations between an open number of partners who may have 
conflicting interests and who possibly don't know each other when the communication
starts. For this area, known as {\em open telecooperation}, 
advanced security techniques are a necessity. 

Security is needed by both users of computer systems and providers of computer
services. The classical way of achieving security reflects more to the interests
of system providers than to the interests of users.
First, users are often rigidly restricted to services and system resources 
which they have paid for or which 
system administrators and service providers believe to be sufficient for the user's needs,
second, access to system resources and information is provided by authentication
techniques based on passwords which protect, in the best case, the system
providers and the users among each other, but which do not protect users from
privileged personnel, 
and third, the user's activities are often traced as comprehensively as it is necessary
for the system administrator's needs. System providers and system users, however,
have different interests. Central user identification, service restrictions and
activity traces help the providers to protect their systems and to get their payment,
but they limit the comfort of users and, more seriously, they jeopardize the privacy
of users. This leads to closed systems. On the other hand, open systems which do
not provide the possibility to users to set up closed subenvironments are unusable
for a lot of applications.
Open systems in which users are able to demand services, 
in the ideal case, spontaneously, unlimited, if desired anonymously, without
prior registration and without succeeding system trace, cannot be realized by 
the conventional password oriented means.

\subsection{Key Issues of Security}
We can observe basicly four types of threats against communication security
in open networks which can be qualified as follows:
\bi
\m Unauthorized release of information by passive observation.
\m Unauthorized modification of information, i.e. changing, duplicating
   or replaying exchanged information between two entities.
\m Masquerading in various forms.
\m Repudiation of communication, both by the originator or the
   recipient of communication information.
\ei

{\em Authenticity} is the key issue of security. Authenticity of the subjects (persons,
organizational instances, program instances, hardware components) 
and of the objects (information, files, programs,
keys) of information processing systems is the basis where the transferability
of responsibilities relies upon, which in turn is the basis of every kind
of cooperative work. Security requirements as integrity of data, access control,
non repudiation or prevention of masquerade (i.e. playing the role of others)
can only be served if one is able to rely on authentic partner relations.

{\em Confidentiality} is an issue, too, for a number of applications.
This is the classical playground of cryptography. Information which belongs
to the privacy of people like personal data, medical data etc., 
or information which represent an economic asset for individuals or organizations,
should not be transmitted in cleartext over public
networks if the disclosure of such information by unauthorized people
could lead to economic or other damage. 
A new dimension of eavesdropping threats comes through the use
of local area networks. Systems which are connected to the entry points
of public data networks are rarely single computers which can be additionally
protected by physical conditions. In most cases
local area networks or distributed systems are behind the wide area network.
An Ethernet for instance is extremely easy to monitor if one has access to
the coax cable, and the extensive use of distributed file systems like NFS
leads to the situation that often one even doesn't know what heavy
network activities are triggered by a command which looks like a local
file access at the operating system interface. This example also shows
that standard measures against eavesdropping as line encryption
wouldn't help here. End-to-end encryption between the connected applications 
is essential in many cases, while encryption measures betwen network
components are additional possibilities which might be useful to
encounter traffic analysis and the like.

However, it should be noted that achieving confidentiality is not possible without
authenticity. One must be sure with whom to exchange or share keys for confidentiality
purposes, otherwise encryption might not be of great value. Authenticity is a
prerequisite for other security services and therefore the key issue of security.

To summarize, we can observe a need for
\bi
\m the possibility of authenticly identifying acting persons or instances in 
   a decentralized and user controlled way which requires the disclosure of
   not more individual information than necessary in order to satisfy the legitimate
   interests of all concerned parties, and which avoids any unnecessary involvement
   and traces of central authorities,
\m the possibility of providing and verifying a proof of authenticity and integrity
   of information,
\m and the possibility of guaranteeing confidentiality and privacy in
   a multi-party environment.
\ei

\subsection{Cryptography}
The basic means to provide security of the nature mentioned above is cryptography.
Cryptographic algorithms are being used for two purposes. One is to prove to others that
one is in the possession of a certain key. If the partner or verifier is able to decrypt
an information which was previously encrypted using this key, the proof is provided.
If it is assured by other means, in addition, that no other person or instance can
be in possession of this key, the use of this key produces a link to the user of the key.
This is the way authentication is done. The other is to use cryptography in order
to conceal information, i.e. to avoid unauthorized disclosure of information.
This is the classical application of cryptography which aimes at achieving confidentiality. 

We can distinguish two main categories of cryptographic algorithms. One category
comprises the {\bf symmetric} algorithms where encryption and corresponding decryption is performed 
using the same key, but inverse functions. The best known symmetric algorithm is the
{\em Data Encryption Standard} (DES) invented 1977 by the American NBS (National Bureau of
Standards, now NIST) and internationally standardized as DEA1 (Data Encryption Algorithm 1).

The second category comprises the {\bf asymmetric} algorithms. This type of algorithms was invented by Diffie and Hellmann in 1976 
and uses different keys for encryption
and corresponding decryption. The two keys are mathematically dependent
from each other, but it is a requirement to the algorithm that one key can be computed 
from the other key at most in one direction, while the other direction must be 
computationally unfeasable. This property allows to make one key 
(the one which can be derived from the other) public while the other key must be kept secret
and unbreakably linked to a person or instance. Therefore this type of algorithm is also
called {\bf public key} algorithm. 

The class of asymmetric algorithms can be subdivided into the two categories {\bf reversible}
public key algorithms and {\bf irreversible} public key algorithms.  Reversible algorithms
have also the property that applying the encryption function to a cleartext followed by 
applying the decryption function to the ciphertext results in the cleartext again and has
the same effect as doing it in the reverse order, i. e. we have
\begin{center}
$d(e(cleartext, E), D) = e(d(cleartext, D), E) = cleartext$
\end{center}
{\small
\bvtab
\1 with \3 e: encryption function \7 E: encryption key \\
\4         d: decryption function \7 D: decryption key
\evtab
}
This property allows to use a single algorithm and even a single key pair for both
authentication/integrity and confidentiality puposes in that you use your own secret
key to produce a digital signature which can be verified by everyone through your
public key, and that another person uses your public key to send you information
in a confidential way which you can decrypt using your secret key. 
The best knowm reversible asymmetric algorithm
is the {\em RSA} algorithm invented by R. Rivest, A. Shamir and L. Adleman in 1978.

Irreversible asymmetric algorithms do not have this property, i.e it is not possible
with them to recover the cleartext from the ciphertext, and they don't have encryption / decryption
functions in that sense. With them it is possible to verify with a public key component that
a digital signature was produced with the corresponding secret key component. Therefore this type 
of algorithm is also called {\bf signature-only} algorithm. The best knowm signature-only algorithms
are the {\em ElGamal} algorithm invented by T. Elgamal in 1985, and the NIST variant 
of this algorithm, the {\em DSA} algorithm, used for the proposed US FIPS {\em Digital Signature Standard} (DSS).  

Symmetric algorithms can be realized very efficiently, but they have the problem of requiring
a shared secret. Two partners using symmetric cryptography must have the {\em same} secret
key. This makes symmetric algorithms unsuitable for proving individual identities to 
third parties because at least two share the same key and the knowledge of the key doesn't
provide a link to an individual user. Another problem is the necessity to transmit
symmetric keys accross networks if the communcating partners are on different places.
Symmetric cryptography can be used for the purpose of authentication
in conjunction with a central authentication server whom anyone trusts and who keeps the secrets of
all partners.

The problem of the shared secret does not exist with asymmteric cryptography. Every person
or instance, i.e. every subject,  owns a unique {\em pair} of keys. One of them 
is to be published, while the access to the other must be unbreakably restricted to the
subject itself through a secure local environment, e.g. a smartcard environment.
By means of {\em digital signatures}, which are being generated through encrypting
a hash value of the information to be signed using the secret key, and which can be verified by everyone through decrypting the signature using the public key, authentic identifications 
and proofs of integrity can be exchanged in open systems. 
Confidentiality can be achieved, too, in the case of reversible algorithms (RSA) by encrypting 
the information with the recipient's public
key. This can be done by everyone, but only the recipient is able to decrypt
the information with his secret key. 

The application of asymmetric cryptography leads to a number of practical problems, too.
One of them is that most known asymmetric algorithms rely on
complex mathematical problems of number theory and the usage of very large integer numbers,
which makes them considerably slower than DES, for instance. Thus they are not suitable
for the encryption of large amounts of data. A widely accepted solution is to encrypt
data (for the purpose of confidentiality) using symmetric algorithms and to exchange
the corresponding symmetric keys in encrypted form using reversible asymmetric algorithms.

\subsection{Public Key Certification}
Another fundamental problem, which arises in open environments with large numbers of
partners who do not know each other, is the question of the authenticity of public
keys. At the time of verifying a digital signature the verifier must be sure that the 
public key which he uses for the verification of the signature is the public key of 
the supposed signer, i.e. he needs a key-to-name binding which he can trust. 
Without additional measures each user would have to perform an out-of-band verification
of the authenticity of each partner's public key before trusting it. The complexity
of this problem can be reduced by {\em certifying} public keys through a third party whom
both signer and verifier trust.
This third party, the so called {\em certification authority} (CA),
signs the user's public key and his name (plus some additional
data like a period of validity) with its own secret key. This piece of data, signed  
by the certification authority, is called a {\em certificate}. The certificate can be 
verified with the public key of the certification authority. Now two partners can 
authenticate each other by first verifying the partner's digital signature with the 
partner's public key and then verifying the authenticity of the partner's public
key through verifying the digital signature of the certificate using the public key
of the certification authority. Only the public key of the 
certification authority must be trusted, thus reducing the number of public keys 
which the individual user has to trust to one. 

In large user populations comprising perhaps millions of partners, a single certification 
authority is not sufficient. The public keys of certification authorities may be 
certified again by other certification authorities. One can imagine tree structures 
of certification authorities, or net structures in that CAs belonging to different trees
cross-certify each other, thus providing certification pathes or chains of trust 
between individual partners. 

\subsubsection{Public Root Key}
However, the chain of certificates cannot be endless, and the public key in the last 
certificate remains uncertified. This is called the {\em Public Root Key} which the 
user simply has
to trust finally. The authenticity and integrity of this key has to be provided by 
out-of-band means. The key can be published,  for instance, in a human readable form, 
and particular software may enable the user to compare this key with that what is stored
in his computer. Once the public root key has been entered into the local computer
system, its integrity must be protected by local means (for instance
through the use of smartcards). The public root key, though public, is as security
sensitive as the personal secret key. They are the two ends of the security
chain between two individual partners.

\subsubsection{Certificate Revocation}
A certification authority must be able to revoke a certificate which it issued
prior to its expiration time. There may be many reasons for a certification authority
to revoke a certificate:
\bi
\m The user's secret key is assumed to be compromised whereby the
corresponding public component is invalidated.
\m The user's affiliation has changed whereby the distinguished
name contained in the certificate's "subject" field is invalidated.
\m The user is no longer to be certified by the CA.
\m The CA's certificate is assumed to be compromised.
\m The user has violated the CA's security policy.
\ei

A certification authority can mark a certificate which it issued as "invalid" by adding
it to the list of revoked certificates.

Information relative to certificate revocation is propagated by
means of revocation lists, so-called "black lists".
Revocation lists must be made publicly available, for instance by being placed in
the public X.500 Directory.
