\subsection{Zoning, Text Transformations}
\nm{3X}{aux\_enchex}{Zoning, Text Transformations}
\label{aux_encdec}
\hl{Synopsis}

\#include $<$secure.h$>$
 
OctetString  {\bf *aux\_enchex} {\em (text)} \\
OctetString {\em *text};

OctetString  {\bf *aux\_dechex} {\em (hexa)} \\
OctetString {\em *hexa};

OctetString  {\bf *aux\_encap} {\em (text)} \\
OctetString {\em *text};

OctetString  {\bf *aux\_decap} {\em (ap)} \\
OctetString {\em *ap};

OctetString  {\bf *aux\_encrfc} {\em (text)} \\
OctetString {\em *text};

OctetString  {\bf *aux\_decrfc} {\em (pren)} \\
OctetString {\em *pren};

OctetString  {\bf *aux\_rfc2hex} {\em (pren)} \\
OctetString {\em *pren};

OctetString  {\bf *aux\_hex2rfc} {\em (hexa)} \\
OctetString {\em *hexa};

OctetString  {\bf *aux\_hex2ap} {\em (hexa)} \\
OctetString {\em *hexa};

OctetString  {\bf *aux\_ap2hex} {\em (ap)} \\
OctetString {\em *ap};

OctetString  {\bf *aux\_rfc2ap} {\em (pren)} \\
OctetString {\em *pren};

OctetString  {\bf *aux\_ap2rfc} {\em (ap)} \\
OctetString {\em *ap};
\hl{Description}
All zoning and transformation functions transform the character string
of the input OctetString
and return the result in a newly created OctetString which
can be freed by aux\_free\_OctetString().

{\bf aux\_enchex} encodes the content of {\em text}
into a hexadecimal {\em 0--9 A--F} representation.
The returned OctetString has the length
$(text$\pf $noctets)*2$ .

{\bf aux\_dechex} decodes the hexadecimal {\em 0--9 A--F}
representation in {\em hexa}
into the original octets.
The returned OctetString has the length
$(hexa$\pf $noctets)/2$ .

{\bf aux\_encap} encodes the content of {\em text}
into a hexadecimal {\em A--P} representation.
The returned OctetString has the length
$(text$\pf $noctets)*2$ .

{\bf aux\_decap} decodes the hexadecimal {\em A--P}
representation in {\em ap}
into the original octets.
The returned OctetString has the length
$(ap$\pf $noctets)/2$ .

{\bf aux\_encrfc} encodes the content of {\em text}
according to the encoding rules of RFC 1421
resulting in a ``printable encoding''. \\
The length of the returned OctetString is at most
$(text$\pf $noctets)*4/3+4$ .

{\bf aux\_decrfc} decodes the ``printable encoding'' in {\em pren}
according to the encoding rules of RFC 1421. \\
The length of the returned OctetString is at most
$(pren$\pf $noctets)*3/4+1$ .

{\bf aux\_rfc2hex, aux\_hex2rfc, aux\_hex2ap, aux\_ap2hex,
aux\_rfc2ap,} and {\bf aux\_ap2rfc}
transform between the three different printable encodings of
octet strings. ``2'' stands for ``to'' and gives the direction
of the transformation, e.g.
{\em aux\_hex2rfc} transforms the hexadecimal {\em 0--9, A--F}
representation of two printable characters per one octet
into the printable encoding according to RFC 1421 of
four printable characters per three octets. Etc.

\subsection{OctetString Handling}
\nm{3X}{aux\_octetstring}{OctetString Handling}
\label{aux_ostringhand}
\hl{Synopsis}

\#include $<$secure.h$>$
 
OctetString {\bf *aux\_create\_OctetString} {\em (text)} \\
char {\em *text};

OctetString {\bf *aux\_new\_OctetString} {\em (length)} \\
int {\em length};

void {\bf aux\_free\_OctetString} {\em (ostr)} \\
OctetString {\em **ostr};

RC {\bf aux\_append\_OctetString} {\em (to, append)} \\
OctetString {\em *to, *append};

RC {\bf aux\_append\_char} {\em (to, append)} \\
OctetString {\em *to};    \\
char {\em *append};

RC {\bf aux\_searchitem} {\em (ostr, item, pos)} \\
OctetString {\em *ostr};    \\
char {\em *item};    \\
unsigned int {\em *pos};

RC {\bf *aux\_cmpitem} {\em (msgbuf, item, pos)}  \\
OctetString {\em *msgbuf};    \\
char {\em *item};    \\
int {\em pos};

OctetString {\bf *aux\_file2OctetString} {\em (filename)} \\
char {\em *filename};

int {\bf aux\_OctetString2file} {\em (ostr, filename[, flag])} \\
OctetString {\em *ostr};    \\
char {\em *filename};
int {\em flag};
\hl{Description}
{\bf aux\_create\_OctetString} creates an OctetString with length {\em strlen(text)} $+$ 1
and character string {\em text}.
hex'00' denotes end of {\em text}.

{\bf aux\_new\_OctetString} creates an OctetString with length {\em length}. The octets
are set to zero.

{\bf aux\_free\_OctetString} frees both 
{\em *ostr}\pf octets and the struct OctetString {\em **ostr} itself.
The pointer {\em *ostr} is set to NULL on return.
See also paragraph \ref{aux_free} for all aux\_free* functions.

{\bf aux\_append\_OctetString} appends the octets of {\em append} to the octets
of {\em to}
and adds {\em append\pf noctets} to {\em to\pf noctets}.
{\em append} remains unchanged.
If {\em to} doesn't exist or the allocation of new space fails,
a negative integer is returned, otherwise 0.
If {\em append} doesn't exist or has length 0, {\em to}
remains unchanged, too, and the returncode is also 0.

{\bf aux\_append\_char} appends the character string {\em append} to the octets
of {\em to} and adds {\em strlen(append)} to {\em to\pf noctets}.
{\em append} remains unchanged.
hex'00' denotes end of {\em append}.
If {\em to} doesn't exist or the allocation of new space fails,
a negative integer is returned, otherwise 0.
If {\em append} doesn't exist or has length 0, {\em to}
remains unchanged, too, and the returncode is also 0.

{\bf aux\_searchitem}
searches in the OctetString {\em ostr} for a character string equal to
{\em item}. The search begins at position {\em *pos}.
If the search is successful, {\em *pos} gives the position
of the first octet behind the first
{\em item} found in {\em ostr}, and the return code is 0.
If the search is not successful, -1 is returned.

{\bf aux\_cmpitem} returns 0, if the sequence of octets in {\em msgbuf}
starting at position {\em pos} and ending at position
$pos+strlen(item)$
is equal to the sequence of characters in {\em item}.
aux\_cmpitem returns 1, if they are not equal.

{\bf aux\_file2OctetString} reads the file {\em filename} into
an OctetString, which it creates and returns. If {\em filename}
is NULL or has zero length, it reads from stdin.
On error it returns the NULL pointer.

{\bf aux\_OctetString2file} writes the content of the OctetString
{\em ostr} into the file {\em filename},
which it opens according to {\em flag}:
 
0: no overwrite (file must not exist), \\
1: overwrite (file must exist), \\
2: create or overwrite, \\
3: append (file must exist), \\
4: create or append.
 
If {\em filename} is NULL or has zero length, it writes
to stdout ({\em flag} has no meaning in this case).
 
Returncode 0 indicates success, -1 indicates an error.

\subsection{Printing Af-Objects, Sec-Objects and Pem-Objects}
\nm{3X}{aux\_fprint}{Printing af-objects, sec-objects and pem-objects}
\label{aux_printostring}
\hl{Synopsis}

\#include $<$stdio.h$>$ \\
\#include $<$secure.h$>$ \\ [1em]
void {\bf aux\_fprint\_OctetString} {\em (file, ostr)} \\
FILE {\em *file}; \\
OctetString {\em *ostr};

void {\bf aux\_fprint\_BitString} {\em (file, bstr)} \\
FILE {\em *file}; \\
BitString {\em *bstr};

void {\bf aux\_xdump} {\em (buffer, len, addrtype)}  \\
void {\bf aux\_fxdump} {\em (dump\_file, buffer, len, addrtype)}  \\
char {\em *buffer}; \\
int  {\em len, addr\_type}; \\
FILE*{\em dump\_file};

{\bf aux\_fprint\_PSESel} {\em (file, pse)} \\
FILE {\em *file}; \\
PSESel {\em *pse};

{\bf aux\_fprint\_PSEToc} {\em (file, psetoc)} \\
FILE {\em *file}; \\
PSEToc {\em *psetoc};

{\bf aux\_fprint\_KeyBits} {\em (file, keybits)} \\
FILE {\em *file}; \\
KeyBits {\em *keybits};

{\bf aux\_fprint\_KeyInfo} {\em (file, keyinfo)} \\
FILE {\em *file}; \\
KeyInfo {\em *keyinfo};

{\bf aux\_fprint\_AlgId} {\em (file, algid)} \\
FILE {\em *file}; \\
AlgId {\em *algid};

{\bf aux\_fprint\_ObjId} {\em (file, objid)} \\
FILE {\em *file}; \\
ObjId {\em *objid};
\\ [1em]
\#include $<$stdio.h$>$ \\
\#include $<$af.h$>$ \\ [1em]
{\bf aux\_fprint\_ToBeSigned} {\em (file, tobesigned)} \\
FILE {\em *file}; \\
ToBeSigned {\em *tobesigned};

{\bf aux\_fprint\_Certificate} {\em (file, certificate)} \\
FILE {\em *file}; \\
Certificate {\em *certificate};

{\bf aux\_fprint\_CertificateSet} {\em (file, certset)} \\
FILE {\em *file}; \\
SET\_OF\_Certificate {\em *certset};

{\bf aux\_fprint\_Certificates} {\em (file, certificates)} \\
FILE {\em *file}; \\
Certificates {\em *certificates};

{\bf aux\_fprint\_FCPath} {\em (file, fcpath)} \\
FILE {\em *file}; \\
FCPath {\em *fcpath};

{\bf aux\_fprint\_PKRoot} {\em (file, pkroot)} \\
FILE {\em *file}; \\
PKRoot {\em *pkroot};

{\bf aux\_fprint\_PKList} {\em (file, pklist)} \\
FILE {\em *file}; \\
PKList {\em *pklist};

{\bf aux\_fprint\_Crl} {\em (file, crl)} \\
FILE {\em *file}; \\
Crl {\em *crl};

{\bf aux\_fprint\_RevCert} {\em (file, revcert)} \\
FILE {\em *file}; \\
RevCert {\em *revcert};

{\bf aux\_fprint\_PemCrl} {\em (file, pemcrl)} \\
FILE {\em *file}; \\
PemCrl {\em *pemcrl};

{\bf aux\_fprint\_RevCertPem} {\em (file, revcertpem)} \\
FILE {\em *file}; \\
RevCertPem {\em *revcertpem};

{\bf aux\_fprint\_CrlSet} {\em (file, crlset)} \\
FILE {\em *file}; \\
CrlSet {\em *crlset};

{\bf aux\_fprint\_CrlPSE} {\em (file, crlpse)} \\
FILE {\em *file}; \\
CrlPSE {\em *crlpse};

{\bf aux\_fprint\_OCList} {\em (file, oclist)} \\
FILE {\em *file}; \\
OCList {\em *oclist};

{\bf aux\_fprint\_IssuedCertificate} {\em (file, isscert)} \\
FILE {\em *file}; \\
IssuedCertificate {\em *isscert};

{\bf aux\_fprint\_SET\_OF\_IssuedCertificate} {\em (file, isscertset)} \\
FILE {\em *file}; \\
SET\_OF\_IssuedCertificate {\em *isscertset};

{\bf aux\_fprint\_SET\_OF\_Name} {\em (file, nameset)} \\
FILE {\em *file}; \\
SET\_OF\_Name {\em *nameset};

{\bf aux\_fprint\_CertificatePairSet} {\em (file, cpairset)} \\
FILE {\em *file}; \\
SET\_OF\_CertificatePair {\em *cpairset};

{\bf aux\_fprint\_VerificationResult} {\em (file, verifresult)} \\
FILE {\em *file}; \\
VerificationResult {\em *verifresult};
\\ [1em]
\#include $<$stdio.h$>$ \\
\#include $<$pem.h$>$ \\ [1em]
{\bf aux\_fprint\_PemInfo} {\em (file, peminfo)} \\
FILE {\em *file}; \\
PemInfo {\em *peminfo};

\hl{Description}

{\bf aux\_xdump} writes the input {\em buffer} in a
``dump format'' (hexadecimal {\em 0--9, A--F}) to {\em stdout}.
{\bf aux\_fxdump} writes the input {\em buffer} in a
``dump format'' (hexadecimal {\em 0--9, A--F}) to {\em dump\_file}.
If {\em addrtype} is set to 0, xdump will print relative addresses,
otherwise absolute adresses.
{\em len} gives the length of the input {\em buffer}.

{\small
\begin{verbatim}
adress   hexadecimal format                        text
000 000  00000000 00000000  00000000 00000000     |................|
000 010  etc. until end of the buffer (len)
\end{verbatim}
}

The {\bf aux\_fprint\_*} functions print the given structures in a suitable
format to the given file pointer {\em file}. {\bf aux\_fprint\_OctetString}
and {\bf aux\_fprint\_BitString} print a dump-format with relative addresses. 
The print format of KeyInfos
containing RSA keys can be controlled by means of the external variable
\bc
{\small
char print\_keyinfo\_flag;
}
\ec
If this flag is set to ALGID (this is the default), the AlgId of the KeyInfo
is printed. If it is set to DER, the DER code contained in BitString subjectkey  
is printed. If it is set to KEYBITS, the decoded DER code of
subjectkey which shows the two integer components of the key is printed. These three
flags can be OR'ed to get any combination.

\subsection{Copy and Compare}
\nm{3X}{aux\_cpy}{Copy and Compare}
\label{aux_cpy}
\hl{Synopsis}

\#include $<$secure.h$>$
 
ObjId {\bf *aux\_cpy\_ObjId} {\em (oid)} \\
ObjId {\em *oid};

int {\bf aux\_cpy2\_ObjId} {\em (dup\_oid, oid)} \\
ObjId {\em *dup\_oid, *oid};

int {\bf aux\_cmp\_ObjId} {\em (oid1, oid2)} \\
ObjId {\em *oid1, *oid2};

AlgId {\bf *aux\_cpy\_AlgId} {\em (aid)} \\
AlgId {\em *aid};

int {\bf aux\_cpy2\_AlgId} {\em (dup\_aid, aid)} \\
AlgId {\em *dup\_aid, *aid};

int {\bf aux\_cmp\_AlgId} {\em (aid1, aid2)} \\
AlgId {\em *aid1, *aid2};

KeyInfo {\bf *aux\_cpy\_KeyInfo} {\em (keyinfo)} \\
KeyInfo {\em *keyinfo};

int {\bf aux\_cpy2\_KeyInfo} {\em (dup\_keyinfo, keyinfo)} \\
KeyInfo {\em *dup\_keyinfo, *keyinfo};

int {\bf aux\_cmp\_KeyInfo} {\em (keyinfo1, keyinfo2)} \\
KeyInfo {\em *keyinfo1, *keyinfo2};

Certificate {\bf *aux\_cpy\_Certificate} {\em (certificate)} \\
Certificate {\em *certificate};

int {\bf aux\_cpy2\_Certificate} {\em (dup\_certificate, certificate)} \\
Certificate {\em *dup\_certificate, *certificate};

int {\bf aux\_cmp\_DName} {\em  (dname1, dname2)} \\
DName {\em *dname1, *dname2};

DName *{\bf aux\_cpy\_DName} {\em  (dname)} \\
DName {\em *dname};

int {\bf aux\_cpy2\_DName} {\em (dup\_dname, dname)} \\
DName {\em *dup\_dname, *dname};

RevCert *{\bf aux\_cpy\_RevCert} {\em  (revcert)} \\
RevCert {\em *revcert};

Crl *{\bf aux\_cpy\_Crl} {\em  (crl)} \\
Crl {\em *crl};

RevCertPem *{\bf aux\_cpy\_RevCertPem} {\em  (revcertpem)} \\
RevCertPem {\em *revcertpem};

PemCrl *{\bf aux\_cpy\_PemCrl} {\em  (pemcrl)} \\
PemCrl {\em *pemcrl};

CertificatePair *{\bf aux\_cpy\_CertificatePair} {\em  (cpair)} \\
CertificatePair {\em *cpair};

OCList *{\bf aux\_cpy\_OCList} {\em  (oclist)} \\
OCList {\em *oclist};
\hl{Description}
{\bf aux\_cpy\_ObjId} creates a structure ObjId, copies the value
of {\em oid} into it and returns it.
                   
{\bf aux\_cpy2\_ObjId} copies the value of {\em oid} into {\em dup\_oid}
and returns 0 on success, -1 on error.

{\bf aux\_cmp\_ObjId} returns 0 if {\em oid1} and {\em oid2} are equal,
and 1 otherwise.

{\bf aux\_cpy\_AlgId} creates a structure AlgId, copies the value
of {\em aid} into it and returns it.

{\bf aux\_cpy2\_AlgId} copies the value of {\em aid} into {\em dup\_aid}
and returns 0 on success, -1 on error.

{\bf aux\_cmp\_AlgId} returns 0 if {\em aid1} and {\em aid2} are equal
(both in the Object Identifiers and parameters),
and 1 otherwise.

{\bf aux\_cpy\_KeyInfo} creates a structure KeyInfo, copies the value
of {\em keyinfo} into it and returns it.

{\bf aux\_cpy2\_KeyInfo} copies the value of {\em keyinfo} into {\em dup\_keyinfo}
and returns 0 on success, -1 on error.

{\bf aux\_cmp\_KeyInfo} returns 0
if {\em keyinfo1} and {\em keyinfo2} are equal
(in the Object Identifiers, parameters, and BitStrings of keys),
and 1 otherwise.

{\bf aux\_cpy\_Certificate} creates a structure Certificate, copies the value
of {\em certificate} into it and returns it.

{\bf aux\_cpy2\_Certificate} copies the value of {\em certificate} into {\em dup\_certificate}
and returns 0 on success, -1 on error.

{\bf aux\_cmp\_DName} compares the two DName *dname1 and *dname2.
If they represent the same distinguished names,
0 is returned, otherwise 1. \\
The comparison is attribute-wise
and regards upper and lower cases equal.

{\bf aux\_cpy\_DName} copies {\em dname} into a newly generated and
allocated DName structure which is returned. It returns NULLDNAME
if {\em dname} is not a valid DName.

{\bf aux\_cpy2\_DName} copies the value of {\em dname} into {\em dup\_dname}
and returns 0 on success, -1 on error.

{\bf aux\_cpy\_RevCert} creates a structure RevCert, copies the value
of {\em revcert} into it and returns it.

{\bf aux\_cpy\_Crl} creates a structure Crl, copies the value
of {\em crl} into it and returns it.

{\bf aux\_cpy\_RevCertPem} creates a structure RevCertPem, copies the value
of {\em revcertpem} into it and returns it.

{\bf aux\_cpy\_PemCrl} creates a structure PemCrl, copies the value
of {\em pemcrl} into it and returns it.



\subsection{Algorithm and Object Identifiers}
\nm{3X}{aux\_algorithms}{Algorithms and Object Identifiers}
\label{aux_aid}
\hl{Synopsis}

\#include $<$secure.h$>$ 

char {\bf *aux\_ObjId2Name} {\em (oid)} \\
ObjId {\em *oid};

AlgId {\bf *aux\_ObjId2AlgId} {\em (oid)}  \\
ObjId {\em *oid};

ParmType {\bf aux\_ObjId2ParmType} {\em (oid)} \\
ObjId {\em *oid};

AlgType {\bf aux\_ObjId2AlgType} {\em (oid)}  \\
ObjId {\em *oid};

AlgEnc {\bf aux\_ObjId2AlgEnc} {\em (oid)}  \\
ObjId {\em *oid};

AlgHash {\bf aux\_ObjId2AlgHash} {\em (oid)}  \\
ObjId {\em *oid};

AlgMode {\bf aux\_ObjId2AlgMode} {\em (oid)}  \\
ObjId {\em *oid};

AlgSpecial {\bf aux\_ObjId2AlgSpecial} {\em (oid)}  \\
ObjId {\em *oid};

ObjId {\bf *aux\_Name2ObjId} {\em (name)}  \\
char {\em *name};

AlgId {\bf *aux\_Name2AlgId} {\em (name)}  \\
char {\em *name};

ParmType {\bf aux\_Name2ParmType} {\em (name)} \\
char {\em *name};

AlgType {\bf aux\_Name2AlgType} {\em (name)}  \\
char {\em *name};

AlgEnc {\bf aux\_Name2AlgEnc} {\em (name)}  \\
char {\em *name};

AlgHash {\bf aux\_Name2AlgHash} {\em (name)}  \\
char {\em *oid};

\hl{Description}
These routines provide values associated to algorithms in the global
algorithm list {\em alglist} (see SEC(5)).

The parameter type ParmType is one of 
\bi
\m PARM\_NULL, indicating that the algorithm has no parameter,
\m PARM\_INTEGER, indicating a parameter of type INTEGER,
\m PARM\_OctetString, indicating a parameter of type OctetString.
\ei
The algorithm type AlgType is one of                                                   
\bi
\m SYM\_ENC, indicating a symmetric encryption algorithm,
\m ASYM\_ENC, indicating an asymmetric encryption algorithm,
\m HASH, indicating a hash algorithm,
\m SIG, indicating a signature algorithm,
\m OTHER, indicating other type of algorithm.
\ei
The algorithm encryption method AlgEnc is one of 
\bi
\m RSA, indicating RSA algorithm,
\m DES, indicating DES algorithm,
\m DES3, indicating DES3 algorithm,
\m ELGAMAL, indicating ElGamal algorithm,
\m NOENC, indicating other algorithms, e.g. hash algorithms,
\ei
The algorithm hash method AlgHash is one of 
\bi
\m SQMODN
\m MD2
\m MD4
\m MD5
\m SHS
\m NOHASH, indicating other algorithms, e.g. encryption algorithms,
\ei
The algorithm mode AlgMode is one of 
\bi
\m ECB
\m CBC
\m NOMODE, indicating no modes are defined for this algorithm,
\ei
AlgMode is currently used only for DES algorithms.

The algorithm specials AlgSpecial is one of 
\bi
\m WITH\_PADDING
\m WITH\_PEM\_PADDING
\m PKCS\_BT\_01
\m PKCS\_BT\_02
\m NOSPECIAL, indicating no special features are defined for this algorithm,
\ei
The padding features are defiend for DES algorithms. PKCS\_BT\_01 means PKCS\#1
defined block encoding of block type 01. PKCS\_BT\_02 means PKCS\#1 defined block
encoding of block type 02.

Parameter {\em oid} denotes an algorithm's object identifier, parameter
{\em name} denotes an algorithm's name. Valid algorithm names and object 
identifiers can be found in INTRO(3X) under Algorithms.

The above functions map between the indicated parameters.

\subsection{Free Allocated Space}
\nm{3X}{aux\_free}{Free allocated space}
\label{aux_free}
\hl{Synopsis}

\#include $<$secure.h$>$ \\
void {\bf aux\_free\_OctetString} {\em (ostr)} \\
OctetString {\em **ostr};

void {\bf aux\_free\_BitString} {\em (bstr)} \\
BitString {\em **bstr};

void {\bf aux\_free\_ObjId} {\em (oid)} \\
ObjId {\em **oid};

void {\bf aux\_free\_AlgId} {\em (aid)} \\
AlgId {\em **aid};

void {\bf aux\_free2\_AlgId} {\em (aid)} \\
AlgId {\em *aid};

void {\bf aux\_free\_PSEToc} {\em (psetoc)} \\
PSEToc {\em **psetoc};

void {\bf aux\_free2\_PSEToc} {\em (psetoc)} \\
PSEToc {\em *psetoc};

void {\bf aux\_free\_KeyBits} {\em (keybits)} \\
KeyBits {\em **keybits};

void {\bf aux\_free2\_KeyBits} {\em (keybits)} \\
KeyBits {\em *keybits};

void {\bf aux\_free\_KeyInfo} {\em (keyinfo)} \\
KeyInfo {\em **keyinfo};

void {\bf aux\_free2\_KeyInfo} {\em (keyinfo)} \\
KeyInfo {\em *keyinfo};
\\[1em]
\#include $<$af.h$>$ 

void {\bf aux\_free\_ToBeSigned} {\em (tbs)} \\
ToBeSigned {\em **tbs};

void {\bf aux\_free2\_ToBeSigned} {\em (tbs)} \\
ToBeSigned {\em *tbs};

void {\bf aux\_free\_Certificate} {\em (cert)} \\
Certificate {\em **cert};

void {\bf aux\_free2\_Certificate} {\em (cert)} \\
Certificate {\em *cert};

void {\bf aux\_free\_CertificateSet} {\em (certset)} \\
SET\_OF\_Certificate {\em **certset};

void {\bf aux\_free\_CrossCertificates} {\em (ccerts)} \\
CrossCertificates {\em **ccerts};

void {\bf aux\_free2\_CrossCertificates} {\em (ccerts)} \\
CrossCertificates {\em *ccerts};

void {\bf aux\_free\_FCPath} {\em (fcpath)} \\
FCPath {\em **fcpath};

void {\bf aux\_free\_PKRoot} {\em (pkroot)} \\
PKRoot {\em **pkroot};

void {\bf aux\_free\_PKList} {\em (pklist)} \\
PKList {\em **pklist};

void {\bf aux\_free\_Certificates} {\em (certs)} \\
Certificates {\em **certs};

void {\bf aux\_free\_DName} {\em (dname)} \\
DName {\em **dname};

void {\bf aux\_free2\_DName} {\em (dname)} \\
DName {\em *dname};

void {\bf aux\_free\_RevCert} {\em (revcert)} \\
RevCert {\em **revcert};

void {\bf aux\_free2\_RevCert} {\em (revcert)} \\
RevCert {\em *revcert};

void {\bf aux\_free\_Crl} {\em (crl)} \\
Crl {\em **crl};

void {\bf aux\_free2\_Crl} {\em (crl)} \\
Crl {\em *crl};

void {\bf aux\_free\_VerificationResult} {\em (verifresult)} \\
VerificationResult {\em **verifresult};

\hl{Description}
The {\bf aux\_free\_*} {\em (parm)} functions free the respective members of {\em *parm}
and {\em *parm} itself. The pointer {\em *parm} is set to NULL on return.

The {\bf aux\_free2\_*} {\em (parm)} functions free the respective members of {\em parm}
only.

\subsection{ISODE Support}
\nm{3X}{aux\_isode}{ISODE Support}
\label{aux_isode}
\hl{Synopsis}

\#include $<$isode/psap.h$>$ \\
\#include $<$secure.h$>$ \\
\#include $<$IF-types.h$>$ 

OctetString {\bf *aux\_PE2OctetString} {\em  (pe)} \\
PE {\em pe};

PE {\bf aux\_OctetString2PE} {\em (ostr)} \\
OctetString {\em *ostr};
\hl{Description}
{\bf aux\_PE2OctetString}
transforms the ISODE-defined presentation element pe, which
in terms of ISODE represents an ASN.1 octetstring,
into an OctetString, which contains the corresponding ASN.1 transfer
protocol data stream.
On success a pointer to the OctetString is returned.
On error a null pointer is returned.
The structure of an presentation element is defined in $<$isode/psap.h$>$.

{\bf aux\_OctetString2PE}
transforms the OctetString *ostr, which contains an ASN.1 transfer
protocol data stream, into
the corresponding ISODE-defined presentation element pe, which
in terms of ISODE represents an ASN.1 octetstring.
The presentation element is returned.
On error the presentation element is the null element.
The structure of an presentation element is defined in $<$isode/psap.h$>$.

\subsection{Alias Name Mapping}
\nm{3X}{aux\_alias}{Alias Name Mapping}
\label{aux_alias}
\hl{Synopsis}

\#include $<$af.h$>$ 

DName *{\bf aux\_alias2DName} {\em (alias)} \\
char *{\em alias}; 

Name *{\bf aux\_alias2Name} {\em (alias)} \\
char *{\em alias}; 

char *{\bf aux\_DName2alias} {\em (dname, type)} \\
DName *{\em dname}; \\
AliasType {\em type}; 

char *{\bf aux\_Name2alias} {\em (name, type)} \\
Name *{\em name}; \\
AliasType {\em type}; 

char *{\bf aux\_Name2aliasf} {\em (name, type, aliasf)} \\
Name *{\em name}; \\
AliasType {\em type}; \\
AliasFile {\em aliasf};

char *{\bf aux\_DName2aliasf} {\em (dname, type, aliasf)} \\
DName *{\em dname}; \\
AliasType {\em type}; \\
AliasFile {\em aliasf};

Name *{\bf aux\_search\_AliasList} {\em (name, pattern)} \\
Name *{\em name}; \\
char *{\em pattern};

RC *{\bf aux\_add\_alias} {\em (alias, dname, aliasf, prior, writef)} \\
char *{\em alias}; \\ 
DName *{\em dname}; \\
AliasFile {\em aliasf}; \\
Boolean prior; \\
Boolean writef;

RC *{\bf aux\_add\_alias\_name} {\em (alias, name, aliasf, prior, writef)} \\
char *{\em alias}; \\ 
Name *{\em name}; \\
AliasFile {\em aliasf}; \\
Boolean prior; \\
Boolean writef;

RC *{\bf aux\_delete\_alias} {\em (alias, aliasf, writef)} \\
char *{\em alias}; \\
AliasFile {\em aliasf}; \\
Boolean writef;

void {\bf aux\_put\_AliasList} {\em (aliasf)} \\
AliasFile {\em aliasf}; 

Boolean *{\bf aux\_alias} {\em (alias)} \\
char *{\em alias}; 

Boolean *{\bf aux\_alias\_chkfile} {\em (alias, aliasf)} \\
char *{\em alias}; \\
AliasFile {\em aliasf};

Name *{\bf aux\_next\_AliasList} {\em (name)} \\
Name *{\em name};

Name *{\bf aux\_alias\_nxtname} {\em (reset)} \\
Boolean {\em reset};

char *{\bf aux\_alias\_getall} {\em (name)} \\
Name *{\em name};

char *{\bf aux\_fprint\_alias2dname} {\em (file, pattern)} \\
FILE *{\em file}; \\
char *{\em pattern}; 

char *{\bf aux\_fprint\_dname2alias} {\em (file, pattern)} \\
FILE *{\em file}; \\
char *{\em pattern};

AliasList *{\bf d\_AliasList} {\em (ostr)} \\
OctetString *{\em ostr};

OctetString *{\bf e\_AliasList} {\em (alist, aliasf)} \\
AliasList *{\em alist}; \\
AliasFile *{\em aliasf};

void {\bf aux\_free\_AliasList} {\em (alist)} \\
AliasList **{\em alist};

RC {\bf aux\_fprint\_AliasList} {\em (file, alist)} \\
FILE *{\em file}; \\
AliasList *{\em alist};

\hl{Description}
Alias names for distinguished names can be defined on a per user basis in either the PSE-object
AliasList or in the file \$HOME/.af-alias, and on a per system basis in the file
\$TOP/secude/.af-db/.af-alias. If the PSE-object AliasList exists, user-defined aliases
are taken from there, otherwise user-defined aliases are taken from \$HOME/.af-alias.

More than one alias name may be defined for one target name. The target name must be a
distinguished name, either in the Name or in the DName form. The alias names are
stored in an ASN.1 encoded OctetString which decodes into the internal structure
\begin{verbatim}
struct aliaslist {
        struct aliases {
                char          *aname; /* alias names */
                AliasFile  aliasfile; /* SYSTEM or USER Alias File */
                struct aliases *next;
        } *a;
        Name    *dname;               /* distinguished name */
        struct aliaslist *next;
};
typedef struct aliaslist AliasList;
typedef struct aliases Aliases;
\end{verbatim}
If the ASN.1 encoded OctetString is stored in the files \$HOME/.af-alias and \$TOP/secude/.af-db/.af-alias as signed file in order to protect its integrity.
The signature is verified when reading these files and newly generated when
writing these files. If the ASN.1 encoded OctetString is stored as PSE-object
AliasList, the PSE protection mechanism is used to protect its integrity.

{\em type} is an enumerated data type which may have one
of the values ANYALIAS, RFCMAIL, X400MAIL, or LOCALNAME. This allows to distinguish
different types of aliases through their syntax. An RFCMAIL type is an alias name which
contains a '@' character. An X400MAIL type is an alias which contains a '=' character.
A LOCALNAME type is an alias which contains neither a '@' nor a '=' character (a login
name, for instance).
ANYALIAS just refers to the first alias which may be of any type.
AliasFile is an enumerated data type which may have the values useralias
or systemalias. useralias refers to the user alias file, systemalias refers
to the system alias file.

On the first call of a routine which transform aliases to DNames or vice versa, the user 
alias file and the system alias file are read into an internal AliasList which
consists of memory obtained through malloc(). The alias transformation
routines return copies of this AliasList. The AliasList can be released
with {\em aux\_free\_AliasList ((AliasList **)0)}. All Names, DNames and alias names returned by 
the alias routines can be freed with the appropriate free routines. The alias routines 
maintain an internal pointer to the ``current'' entry.  

{\bf aux\_alias2Name} searches for {\em alias} in the AliasList
and returns the corresponding target
(a distinguished name in Name format). If {\em alias}
is not found as alias entry, it returns {\em alias} itself. Since an alias name can be
of the Name form (e.g. in the case of X.400 OR-names being aliases to DNames, see the second 
example above), the comparison of alias names is done on the basis of the DName type to which
the printable Name forms are transformed by {\em af\_alias2Name()}.
Keywords which do not
denote a SecuDE recognized DN attribute, like PRMD and ADMD in the example above,
are ignored for this comparison. This allows to define X.400 OR-names as alias names
for distinguished names.

Since the return value of {\bf aux\_alias2Name} does not allow to distinguish whether
{\em alias} is a defined alias name or not, the function {\bf aux\_alias} can be
used to decide this. {\bf aux\_alias} returns TRUE is {\em alias} is in the AliasList,
otherwise it returns FALSE. {\bf aux\_alias\_chkfile} does the same as aux\_alias
except that it checks only alias from the given alias file.

In the example above, af\_alias2Name("grimm") would return 

"C=DE; O=gmd; OU=I2; CN=R. Grimm".
\\ [1em]
{\bf aux\_alias2DName} returns the corresponding DName if {\em alias} is found in
the alias file. If {\em alias} is not found, 
it returns the DName which is obtained
through transforming {\em alias} itself to the DName type. If this fails, Null is
returned.
\\ [1em]
{\bf aux\_DName2alias} returns the first alias name of type {\em type} given to {\em dname}.
It returns Null if {\em dname} is not found as a target name in the alias list, 
or it has no
alias of the requested type. 
\\ [1em]
{\bf aux\_Name2alias} returns the first alias name of type {\em type} given to {\em name}.
It returns Null if {\em name} is not found as a target name in the alias list, 
or it has no
alias of the requested type. 
\\ [1em]
{\bf aux\_Name2aliasf} is like aux\_Name2alias except that it looks only for aliases
which are defined in the requested AliasFile, i.e. which are either system aliases
or user aliases. 
\\ [1em]
{\bf aux\_DName2aliasf} is like aux\_Name2aliasf except that it handles a DName
instead of a Name
\\ [1em]
{\bf aux\_add\_alias} adds the given {\em alias} and {\em dname} combination to
the alias file indicated by {\em file}. {\em prior} $=$ TRUE, the new alias is
chained to the first position, otherwise it is chained to the last position.
If {\em writef} $=$ TRUE, the corresponding alias file is updated immediately.
If {\em writef} $=$ FALSE, only the internal AliasList is changed. To update
the corresponding alias file, {\bf aux\_put\_AliasList} {\em (file)} must be
used. Aux\_add\_alias returns 0 on success. It returns 0, too, if the
given combination exists already in the alias file. Otherwise, it returns -1.
\\ [1em]
{\bf aux\_add\_alias\_name} does the same as {\bf aux\_add\_alias} except that
it handles a Name instead of a DName.
\\ [1em]
{\bf aux\_delete\_alias} deletes the given {\em alias} from the alias file (either from
the USER alias file or from the SYSTEM alias file, depending where it came from).
{\em writef} has the same meaning as above.
\\ [1em]
{\bf aux\_search\_AliasList} returns a name which contains {\em pattern} in any of
its attributes. If the input parameter {\em name} is a string of length zero,
it searches from the ``current'' position. At the end of the AliasList it starts
again at the begin of the AliasList and searches at most until to the position
which was ``current'' when aux\_search\_AliasList was called. If {\em name} is the 
NULL pointer, ``current'' is set to the begin of the AliasList before the search starts.
If {\em name} is a Name, it starts the search after this position. The AliasList
is sorted in alphabetic order of the target which is a distinguished name in the Name form.
\\ [1em]
{\bf aux\_next\_AliasList} is like aux\_search\_AliasList except that it doesn't searh
for a pattern but returns just the next name. 
\\ [1em]
{\bf aux\_alias\_nxtname} has its own ``current'' pointer which is independent of that
of the other functions. It returns the next name after this ``current'' and increments
its ``current''. In contrast to the other routines it does not wrap over the end of
the AliasList, but returns a NULL pointer in case that the end of the AliasList
is reached and sets its ``current'' to the begin of the AliasList. If{\em reset} is TRUE,
it returns the first name of the AliasList and sets its ``current'' to top.
\\ [1em]
{\bf aux\_alias\_getall} returns all aliases of the given name in one character string.
The aliases are separated by ':' characters.
\\ [1em]
{\bf aux\_fprint\_alias2dname} searches alias names which contain {\em pattern} and
prints one line per hit of the form
\bc
{\em aliasname} - - -$>$ $<${\em distinguished name}$>$
\ec
to file pointer {\em file}.
\\ [1em]
{\bf aux\_fprint\_dname2alias} searches distinguished names which contain {\em pattern} and
prints one line per hit of the form
\bc
$<${\em distinguished name}$>$ - - -$>$  {\em aliasname}
\ec
to file pointer {\em file}.

The following routines are intended for SecuDE-internal use. {\bf d\_AliasList} decodes the
given OctetString {\em ostr} into an AliasList structure. {\bf e\_AliasList} does the reverse
process, whereby it encodes only those aliases of which the source is the same as the
indicated AliasFile {\em aliasf}, i.e. it encodes only the user aliases if aliasf $=$ useralias,
or only the system aliases if aliasf $=$ systemalias. {\bf aux\_free\_AliasList} frees
the given {\em alist}, and {\bf aux\_fprint\_AliasList} prints the given {\em alist}
to the given {\em file}.

\subsection{UTCTime handling Functions}
\nm{3X}{aux\_UTCTime}{UTCTime handling Functions}
\label{aux_time}
\hl{Synopsis}

\#include $<$secure.h$>$

extern Boolean aux\_localtime;

UTCTime *{\bf aux\_current\_UTCTime} {\em ()}

UTCTime *{\bf aux\_delta\_UTCTime} {\em (utctime)} \\
UTCTime *{\em utctime}; 

int {\bf aux\_interval\_UTCTime} {\em (utctime, notbefore, notafter)} \\
UTCTime *{\em utctime}; \\ 
UTCTime *{\em notbefore}; \\ 
UTCTime *{\em notafter};

int {\bf aux\_cmp\_UTCTime} {\em (utctime1, utctime2)} \\
UTCTime *{\em utctime1}; \\
UTCTime *{\em utctime2}; 

char *{\bf aux\_readable\_UTCTime} {\em (utctime)} \\
UTCTime *{\em utctime}; 

\hl{Description}
{\bf aux\_current\_UTCTime} returns the current time in UTCTime format. If
{\em aux\_localtime} is TRUE, the time is generated as local time (plus
or minus the time difference to GMT), otherwise it is generated as GMT.

{\bf aux\_delta\_UTCTime} returns {\em utctime} plus 1 year, or NULL if
{\em utctime} is invalid. The returned UTCTime is in the local time format
if {\em aux\_localtime} is true and in the GMT format otherwise. 

{\bf aux\_interval\_UTCTime} returns 0 if {\em utctime} is inside the interval
between {\em notbefore} and {\em notafter} (including the boundaries), or 1 if
it is outside. It returns -1 if one of the given UTCTimes is invalid.

{\bf aux\_cmp\_UTCTime} returns 1 if {\em utctime1} is later than {\em utctime2}.
It returns 2 if {\em utctime2} is later than {\em utctime1}. It returns 0
if both times are equal. It returns -1 if one of the UTCTimes is invalid.

{\bf aux\_readable\_UTCTime} returns {\em utctime} is printable character string
as produced by {\em ctime()}, however without trailing newline character. It
returns NULL is {\em utctime} is invalid.

Input parameters in UTCTime format can be both in GMT or local time format. The
format of output parameters in UTCTime format is controlled by {\em aux\_localtime}.
Necessary memory is allocated by the called program and can be freed using
{\em free()}.

\subsection{Distinguished Name Handling}
\nm{3X}{aux\_DName}{Distinguished Name Handling}
\label{aux_name}
\hl{Synopsis}

\#include $<$isode/psap.h$>$ \\
\#include $<$secure.h$>$ \\
\#include $<$IF-types.h$>$ 

DName *{\bf aux\_Name2DName} {\em (name)} \\
char {\em *name};

char *{\bf aux\_DName2Name} {\em (dname)} \\
DName {\em *dname};

int {\bf aux\_cmp\_DName} {\em  (dname1, dname2)} \\
DName {\em *dname1, *dname2};

int {\bf aux\_cmp\_Name} {\em  (name1, name2)} \\
Name {\em *name1, *name2};

int {\bf aux\_checkPemDNameSubordination} {\em  (sup, sub)} \\
DName {\em *sup, *sub};

\hl{Description}
{\bf aux\_Name2DName} and {\em aux\_DName2Name} transform between
the printable character string representation {\em Name} and the
C-structure {\em DName} of distinguished names. 

{\bf aux\_cmp\_DName} compares the two DName *dname1 and *dname2.
If they represent the same distinguished names,
0 is returned, otherwise 1. 

{\bf aux\_cmp\_Name} compares the two Name *name1 and *name2 by
transforming the printable character string representation {\em name1} 
and {\em name2} into DName structure representations using
{\em aux\_Name2DName()} and applying {\em aux\_cmp\_DName()}
afterwards.
If they represent the same distinguished names,
0 is returned, otherwise 1.

{\bf aux\_checkPemDNameSubordination} compares the two DName structures {\em *sup} and {\em *sub}.
1 is returned, 
if {\em sub} is subordinate to {\em sup} in the DIT, i.e., {\em sub} is constructed of exactly the same
Relative Distinguished Name components as {\em sup}, with some additional Relative Distinguished Name
components appended.
If subordination is not detected, 0 is returned. -1 is returned in case of invalid parameters.\\
This routine is useful in the context of the PEM trust model (RFC 1422), where 
Organizational CAs are expected to sign certificates only if the subject distinguished
name in the certificate is subordinate to their own distinguished name, thereby 
documenting the subject's affiliation to the organization.

\subsection{Error Handling}
\nm{3X}{aux\_error}{Error Handling}
\label{aux_error}
\hl{Synopsis}

\#include $<$af.h$>$ 


extern struct ErrStack *err\_stack; 


void {\bf aux\_add\_error} {\em (number, text, addr, addrtypr, proc)} \\
int {\em number}; \\
char {\em *text, *addr}; \\
Struct\_No {\em addrtype}; \\
char {\em *proc}; \\ [1em]
void {\bf aux\_fprint\_error} {\em (file, last\_only)} \\
FILE {\em *file}; \\
int {\em last\_only}; \\ [1em]
void {\bf aux\_free\_error} {\em ()} \\ [1em]
int {\bf aux\_last\_error} {\em ()}
\hl{Description}
For error reporting purposes, SecuDE provides an error stack which collects all
errors which occured during the processing of Secude functions. This stack is not
intended to be exploited by normal users, but for the experienced user or programmer,
who is rougly familiar with the program code, it provides some backtracing
functionality in case of errors. The stack can be lengthy depending on the depth
of nested function calls.  

The error stack has the following form: 

\bvtab
\2 typedef struct \{ \\
\4        int  \4            e\_number; \\
\4        char \4           *e\_text;   \\
\4        char \4           *e\_addr;   \\
\4        Struct\_No   \4     e\_addrtype; \\
\4        char \4           *e\_proc;   \\
\4        struct ErrStack  \4 *next;   \\
 \2 \} ErrStack; \\ 
\evtab

The error stack is ordered most recent first.
 
The global variable {\em err\_stack}
is the pointer to the error stack. $err\_stack = NULL$ indicates that no error occured so far.

The error stack holds the following information per entry:

{\em e\_number} is the corresponding error number (see INTRO(3)), {\em e\_text} is an additional
explanatory text. {\em addr} can be either NULL or the address of a data element (i. e. 
structure address) which is interesting in conjunction with the error. {\em addrtype} is the
type of {\em addr} (see secure.h for appropriate values). {\em proc} is the name of the 
routine where the error occured.
\\ [1em]
{\em aux\_add\_error()} adds an entry to the error stack. It allocates the necessary memory.
\\ [1em]
{\em aux\_fprint\_error()} prints the current error stack in suitable form to file pointer {\em file}.
If {\em last\_only} is not NULL, only the most recent entry is printed.
\\ [1em]
{\em aux\_free\_error()} frees the error stack and sets err\_stack to NULL.
\\ [1em]
{\em aux\_last\_error()} returns the error number of the most recent error, or 0 if
no error occured.

