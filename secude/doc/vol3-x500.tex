\section{Secure X.500 Directory}
\markboth{Directory Services}{Directory Services}
\thispagestyle{myheadings}
\label{ds}
\subsection{System Description}
The X.500 Directory is an OSI application which has been developed
jointly by CCITT and ISO. The CCITT version is specified in the
X.500 series of Recommendations; the ISO version is known as
International Standard 9594.

The directory is a system holding information about various entities
(referred to as {\em objects}) and providing users with services for
accessing the information. Examples of entities likely to be
represented in the directory are countries, organizations,
organizational subdivisions, people, etc.
The information held on a particular person may include the
person's name, postal address, telephone number, and electronic mail
address; the information about a particular organization may
comprise that organization's name, address, telephone number, and
telex number.
The information held in the directory is used to facilitate communication
between entities.
The complete set of information to which the directory provides
access is known as the {\em Directory Information Base} (DIB).
\\ [1em]
The directory can be used to support the following services:
\\ [1em]
{\bf Look-up}

The directory supports human users in obtaining, among other things,
telephone and address information of organizations and other users.
\\ [1em]
{\bf Message Handling Support}

The directory can be used in message handling (based on CCITT's
X.400 standard) where it provides the mapping of user-friendly names
(supplied to the directory) onto machine oriented O/R addresses
(returned from the directory). The directory also provides the means
by which two MHS functional entities (two MTAs, or a UA and a MTA)
may establish the identity of each other.
\\ [1em]
{\bf Support of Authentication Process}

The directory meets the needs for authentication and other security
services (e.g. content confidentiality, content integrity) as it is the
place from where communicating peers can obtain
knowledge on each other. The directory provides two approaches to
authentication:
\bi
\m The {\em simple authentication} approach relies on the directory holding
a password for any user that wishes to be able to authenticate itself
to a service.
\m The {\em strong authentication} approach is based on public-key
cryptography; here the directory acts as a repository of {\em public keys}
within {\em certificates}.
\ei
A certificate can be revoked by being added to a
{\em revocation list}.
It is of vital importance that certificates and revocation lists be
easily available to a distributed community; this is achieved by placing
them in a X.500 Directory.
\\ [1ex]
The need for authentication may be uni- or bilateral: In the first case
a user may wish to authenticate itself to an application in order
to be permitted to carry out some action; in the latter case two
communicating entities (applications and/or users) may wish to establish
the identity of each other.
(The directory itself requires its users to
authenticate themselves before enabling them to access confidential
information or to modify some of the directory information.)


\subsubsection{The Directory Model}

The directory is comprised of a set of one or more application
processes known as {\em Directory System Agents} (DSAs). In case
of the directory being composed of more than one DSA, it is said to
be {\em distributed}. In a distributed directory each DSA holds
only a fragment of the total directory information.
\\ [1ex]
Each DSA provides zero, one, or more {\em access points} to the directory
at which users may interact with the directory. A user accesses the
directory by means of an application process known as a {\em Directory
User Agent} (DUA), running in the user's system.
The interaction between a DUA and a DSA is carried out by means of a
directory protocol called {\em Directory Access Protocol} (DAP).
\\ [1ex]
A basic feature of a distributed directory is that a user
should potentially be able to access any directory information
(subject to access control) irrespective of the access point at which
his request originates and without knowledge of the physical location
of the required information. It is required that the distribution
of the directory information among multiple DSAs be hidden from the
user, thereby giving the user the impression that each DSA holds
the whole of the DIB. This is achieved by each DSA (holding
a fragment of the DIB) having some knowledge about the location
of the rest of the DIB.
\\ [1ex]
As a consequence, a DUA need only have direct access to a single DSA to
which it can submit the user's request: If that DSA cannot
satisfy the request, it will be able to locate the DSA holding
the requested information and either return that knowledge to the
requestor or pass on the request to the appropriate DSA. (The former
mode of interaction is known as {\em referral}, the latter one
is known as {\em chaining}.) The interaction between two DSAs is
defined by the {\em Directory System Protocol} (DSP).


\subsubsection{The Information Model}

The information model describes the logical structure of the DIB; in
this context the physical distribution of the DIB across multiple DSAs
is not considered.

The DIB is composed of {\em entries}, each representsing one
(directory) object and containing the information on that object.
An entry is structured as a set of {\em attributes}, each with an
{\em attribute type} and one or more {\em attribute values}.

The DIB is hierarchically structured by arranging its entries
in a tree, known as the {\em Directory Information Tree} (DIT).
The DIT reflects the hierarchical relationship commonly found
among objects. For example, a person works for a department,
which belongs to an organization, which has its head office
in some country. As a consequence, objects representing countries
will be found close to the root of the DIT, whereas entries representing
people or application processes will be located in the leaves of the DIT.

The DIT provides a means of assigning each entry a {\em Distinguished
Name} (DN) which unambiguously identifies that entry; this means that
no two entries may have the same DN.
Naming is accomplished by assigning a {\em Relative Distinguished Name}
(RDN) to each entry. The RDN of an entry is made up of specially
nominated attribute values (the {\em distinguished values}) of the
entry.

The DN of an entry is then defined as being the sequence of RDNs of the
entry and of all its superiors in descending order from the root.

The second hierarchy which rules the DIB is that of the object classes.
An object class represents a family of objects which share certain
characteristics. Every object belongs to at least one object class.
An object class defines which attributes must be present ({\em mandatory}
attributes) and which attributes may be present ({\em optional}
attributes) in an entry of the object class.

Example:

The object class {\em person} defines the attributes {\em commonName}
and {\em surname} as being mandatory and the attributes {\em description},
{\em seeAlso}, {\em telephoneNumber}, and {\em userPassword} as being
optional in an entry of that object class.

An object class may be a {\em subclass} of another object class. In
this case the members of the subclass share all the characteristics
of the {\em superclass} and additional characteristics possessed
by none of the members of the superclass. As a consequence, an object
belonging to a particular object class does not only contain
the mandatory attributes of that object class, but also the
mandatory attributes of all the superclasses of that object class.
(The same applies to the optional attributes.)



\subsection{System Security Policy}

The X.500 Directory appears both as user of security services in order to
protect its stored information from unauthorized access and to protect its 
communication, and as part of the security infrastructure in that it provides 
security related information to users which enable them to use security 
services for other applications.

Security policies for Directories described in this chapter emphasize the 
first aspect, the protection of the Directory itself. They aim at three 
major goals:

\be
\m to protect the information base of the Directory,
\m to protect the internal and external communication of the Directory
   via DAP and DSP,
\m to protect the resources of the Directory.
\ee

With SecuDE we strictly rely on standardized security services and 
mechanisms. Therefore, Directory security policies of SecuDE  only 
require services and mechanisms which are defined in X.500 and other 
related standards. The X.500 security model for Directories focuses on 
the protection of the Directory information base when remotely accessing 
to it via DAP and/or DSP. Local access to the DIB, for instance through
DSA managers, and integrity or confidentiality of the information itself 
with respect to the originator of the information, is out of scope of
X.500 defined security features.
 
DSA and DUA capabilities and their communication via DAP and DSP are 
according to X.500-88. X.500-92 access control facilities, for instance, 
are out of scope of SecuDE.


\subsection{Authentication and Authorization Policies}

The Directory information base is distributed over a large number of DSAs,
each of them maintaining a particular part of the DIB. Remote access to
a particular entry of the DIB is done either through direct DUA access
to the serving DSA (which maintains either the master entry or a shadow 
entry) via DAP, or through chained DSA access to the serving DSA via DSP 
on behalf of the requesting DUA.

This operation can be protected in terms of authentication and authorization.
An authentication policy is used to describe how the providers and the users
of Directory services are to be identified each other in order to gain a 
required level of trust between them. Different authentication methods are
defined in X.500. An authorization policy is then used to derive access rights 
from the authenticated identities, in order to grant or deny access to the
requested entry. 

\subsection{Directory Authentication}


\subsubsection{Authentication during Directory Bind or DSABind}
\label{bind}

When a remote entity binds to a DSA, it shall prove its identity to the
DSA by suitable means. The initiating remote entity may be a DUA, or another 
DSA performing chained operations. One of the following authentication
methods may apply:

\bi

\m No Authentication

  The initiator provides only his distinguished name (DN), but no further
  credential. In that case it is impossible to authenticate the initiator's 
  identity, with the effect that this kind of access is equivalent to an
  anonymous access.

\m Simple Unprotected Authentication

  The initiator provides his DN together with a plaintext password to the
  DSA. This authentication method requires a shared secret between initiator 
  and DSA.

\m Simple Protected Authentication

  The initiator provides his DN, a time-stamp, a random number and a message
  digest to the DSA. He generates the message digest by hashing the 
  time-stamp, the random number and his password with a predefined one-way hash 
  function. The DSA authenticates the initiator by doing the same process and 
  comparing his hash result with the provided message digest. This authentication
  method requires a shared secret between initiator and DSA, but avoids the 
  cleartext transmission of the password.

\m Strong Authentication

  The initiator provides a digitally signed security token, optional together
  with a set of certificates comprising the forward certification path, to the 
  DSA. The DSA verifies the digital signature through finally applying a public 
  key which he trusts. This authentication method does not require a shared secret
  between initiator and DSA if the digital signatures are based on asymmetric 
  cryptography. It requires a common point of trust between initiator and DSA and 
  the ability to gain the certification path between them when digital signatures 
  are to be verified.
\ei
In any of the authentication methods it is possible that credentials are
not only passed from the initiator to the responder, but also from the responder
to the initiator, in order to provide mutual authentication between the entities.

X.500 also allows the use of credentials based on an externally-defined and 
bilaterally-agreed scheme.


\subsubsection{Signed Directory Operations}


If an association between initiator and DSA has been established using strong
authentication, the subsequent Directory operations may be digitally signed.
Signed Directory operations consist of
\bi
\m the original unprotected operation,
\m the algorithm identifier of the algorithm used for the digital signature,
\m the signature itself.
\ei
It is possible to indicate in an operation request that the operation result
shall be signed.

If an association between initiator and DSA has been established using simple
or no authentication methods, only subsequent unprotected Directory operations 
are possible.


\subsection{Directory Authorization}
\label{authorization}


Directory authorization is the process of providing/refusing Directory services 
or granting/denying access rights to DIB entries depending on the result of a
preceding authentication of the identity of the requestor. Whenever a Directory
operation is requested either via DAP or DSP, an authorization policy is consulted.
A DSA may refuse any service or Directory operation independent of the identity of 
the requestor and regardless of the established access rights if the method of 
authentication appears inadequate, i.e. it doesn't meet the requirements of the
security policy in force.

X.500-88 provides guidance, but does not provide any mechanism or regulation on
how to enforce authorization. This is a local DSA matter. In order to be able
to carry out security policies appropriately, a DSA should have the capability,
for instance through maintaining access control lists, 
\bi
\m to grant or deny access to a particular entry on the basis of the identity 
  of the requestor (for instance to allow access to particular attributes only
  if the name of the requestor matches certain conditions),
\m to grant or deny access to a particular entry on the basis of the method of 
  authentication of the requestor(for instance to allow access to particular 
  attributes only if strong authentication has been used),
\m to provide or refuse a particular type of operation on the basis of the identity
  of the requestor (for instance to allow modify operations only to the owner
  of the entry),
\m to provide or refuse a particular type of operation on the basis of the method
  of authentication of the requestor (for instance to allow modify operations
  only if strong authentication has been used),
\m to provide or refuse service on the basis of the identity of the requestor (for
  instance if a DSA and the part of the DIB it maintains only serve  to
  a closed group of users),
\m to provide or refuse service on the basis of the method of authentication
  of the requestor (for instance to restrict any access to a part of the DIB
  to strongly authenticated users only).
\ei
A DSA should be able to handle any combination of the above conditions correctly.


\subsection{Distributed Directory Operations}


The enforcement of the necessary service control or access control measures
depending on the security policy is easy if the DSA which is accessed by a user
via DAP holds the requested DIB entry, i.e. if no chained DSP operations are
necessary.

If this is not the case, i.e. if the DSA has to chain the request via DSP to
another DSA, things are more complicated. In this case we can distinguish 
four different types of entities which are involved in the Directory operation:
\be
\m The originating DUA. This is the user who requests the Directory operation.
   He approaches a DSA and offers one of the authentication credentials listed
   in~\ref{bind} during the DirectoryBind.

\m The authenticating DSA. This is the DSA which is approached by the requesting 
   DUA via DAP. This DSA has to authenticate the requesting user during the 
   DirectoryBind, i.e. it has to accept or refuse the DAP association, and then, 
   in the positive case, to initiate the chained operation to another DSA via DSP. 
 
   In case of simple unprotected authentication methods being applied, the 
   authenticating DSA will initiate a chainedCompare operation to the DSA which 
   holds the master entry of the requesting user for the purpose of authentication
   during DirectoryBind.

   In case of simple protected authentication methods being applied, the 
   authenticating DSA will initiate a chainedCompare operation to the DSA which 
   holds the master entry of the requesting user, too. However, defect reports 
   have shown that the current X.500 standard is deficient in that respect and that 
   additional definitions have to be introduced in order to be able to perform simple 
   protected authentication in the distributed case.

   In case of strong authentication methods being applied, the authenticating DSA
   will verify itself the digital signature of the requesting user without the
   help of other DSAs. 
   
\m Intermediate DSAs. They neither authenticate the requesting user nor satisfy 
   the Directory operation. Distributed Directory operations do not necessarily
   involve intermediate DSAs.

\m The serving DSA. This is the DSA which finally satisfies the Directory operation, 
   i.e. the DSA which holds the requested entry of the DIB. This can be a DSA which 
   holds the master entry or one which holds a shadow entry.
\ee

\subsubsection{DSA - DSA Authentication}


When performing the chained DSP operation, the DSAs involved  may also authenticate
each other during DSABind, applying one of the authentication methods listed in 
\ref{bind}.

\subsubsection{Signed Distributed Directory Operations}


In addition to authentication during DirectoryBind and DSABind, each distributed 
Directory operation may be optionally signed. Two signatures may be involved:
\be
\m The signature of the original requestor is transferred through all chaining DSAs 
   to the serving DSA, i.e. the serving DSA is able to verify the signature of 
   the original requestor. This allows the serving DSA to consult its authorization 
   policy and to decide whether or not to perform the requested Directory operation.

\m Each chained operation between two DSAs, which consists of ChainingArguments and 
   the originally requested Directory operation, may be signed by the initiating
   DSA. This is an additional measure to assure that only trusted DSAs are 
   performing distributed Directory operations.
\ee
It is possible to indicate in an operation request that the operation result
shall be signed.


\subsubsection{Navigation Knowledge Information}


While a particular DSA may store only a portion of the DIB, it must have some 
knowledge about the location of the rest of the DIB in order to be able to do 
reasonable chained operations. Proper distributed operation of the Directory as 
a whole relies on carefully maintained and uncompromised navigation knowledge 
information. If navigation information is altered by unauthorized users,
the provision of reliable Directory services can be seriously jeopardized.
Functionality, formats and operations on navigation knowledge information,
however, is not standardized by X.500. Local implementation measures are to
be used to provide a high degree of security in this area.


\subsubsection{Some Consequences on Security Policies}


The X.500 DSP does not provide any means of transferring the knowledge about the
method of authentication between the originating DUA and the authenticating DSA
(during DirectoryBind) to the serving DSA. It also does not provide any means
of transferring the name of the authenticating DSA to the serving DSA. It does provide,
however, the possibility of transferring the name of the original requestor to the
serving DSA. This leads to the following consequences:
\bi
\m If a security policy requires the use of strong authentication in order to
  allow a particular Directory operation, signed Directory operations must be
  used. Only in this case is the serving DSA  in a position to know that strong
  authentication has been used and to verify the signature of the original 
  requestor. Signed Directory operations, in turn, require a strong authentication
  DirectoryBind between DUA and authenticating DSA. If a serving DSA receives an 
  unsigned Directory operation, it must assume that the method of authentication 
  is no better than simple.

\m An authenticating DSA should insert the name of the original requestor into
  the chained operation only if the method of authentication is at least simple.
  If the original requestor did not provide any credential, this name should
  remain empty. Thus, a serving DSA
\bi
\m may assume strong authentication only if it receives a signed Directory
    operation,
\m may assume simple (unprotected or protected) authentication if it receives
    an unsigned Directory operation which contains the name of the original
    requestor,
\m must assume that no authentication has been used if it receives an unsigned
    Directory operation which does not contain the name of the original requestor.
\ei
\ei
\subsection{Objectives of Directory Security Policies}

The objectives of Directory security policies are to ensure the protection of 
particular DSAs and their data information base against a number of threats
while being part of the global Directory. This protection is mainly being
provided through allowing only authorized entities to carry out specific
Directory operations via DAP or DSP, while unauthorized entities are prevented 
from doing so.


\subsubsection{Threats to the Directory}


There are many possible threats to Directory security. Some of them are covered
by the X.500 security model, some of them must be countered by local implementation
and maintainance measures. The following text on threats is taken from 
EWOS/EGDIR/92/266. It is based on Annex A of X.509-88. 

\bi
\m Identity Interception

The identity of one or more of the users involved in a communication is observed
for misuse.

The Directory provides no inherent protection. In particular, note that as use of
the Directory may involve chaining, it is difficult to even bound the end-points
of the communications.


\m Masquerade


A user pretends to be a different user in order to gain access to information or
to acquire additional privileges. This is considered to be the main threat
against Directory services.

Use of simple protected or strong credentials, or signed Directory operations, 
provides protection.


\m Replay


This is the recording and subsequent replay of a communication at some later date
(with the intent of masquerade). The use of simple protected or strong credentials,
combined with signed Directory operations, provides some protection.

 
\m Data Interception


This is the observation of user data during a communication by an unauthorized user.

The Directory provides no inherent protection. However, a data confidentiality 
service could be used to provide this. Further, if strong credentials are used, then
a session key may be exchanged for use with the data confidentiality service. Note
that in the absence of a data confidentiality service, a DSA may refuse to disclose
information or allow security-related information to be modified.


\m Manipulation


This is the replacement, insertion, deletion or misordering of user data during a
communication by an unauthorized user.

Use of strong credentials combined with signed Directory operations provides protection.


\m Repudiation

This is the denial by a user of having participated in part or all of a
communication.

Use of strong credentials with signed Directory operations provides protection.


\m Denial of Service

This is the prevention of interruption of a communication or the delay of
time-critical operations. 

The Directory provides no inherent protection. This topic is outside the scope
of this document. However, competent systems programming often provides adequate
protection against common variants of this threat, e.g, a DSA may refuse
to process queries which it considers too expensive.


\m Mis-Routing


This is the mis-routing of a communication path intended for one user to another.

The Directory provides no inherent protection. However, the use of strong
credentials coupled with signed Directory operations sometimes permits detection
of this threat at the receiving end.


\m Traffic Analysis

This is the observation of information about a communication between users
(e.g. absence/presence, frequency, direction, sequence, type, amount, etc.).

The Directory provides no inherent protection. This topic is outside the scope
of this document. However, low-layer encryption, e.g. at the data link layer,
may provide adequate protection in some environments.


\subsubsection{Countermeasures and Mechanisms}

The primary threat to Directories which can be countered by X.500 defined
security measures is presumed to be that of masquerade. Therefore, we will
support security policies which require DSA capabilities listed in~\ref{authorization}.

In particular, the following DUA and DSA functionality will be provided:
\bi
\m DUAs which are able to to strongly authenticate towards
  a DSA during DirectoryBind, based on user request at the user interface.

\m DUAs which are able to strongly authenticate a DSA when
  processing a DirectoryBindResult.

\m DUAs which are able to perform digitally signed Directory
  operations, both actively and passively, based on user request at the user 
  interface.

\m DSAs which are able to perform the according strong 
  authentication procedures during DirectoryBind via DAP.

\m DSAs which are able to verify Directory operations which are
  digitally signed by the original requestor when acting as serving DSA , and
  which are able to produce digitally signed operation results.

\m DSAs which are able to transfer signed directory operations
  through chained operations, i.e. to transfer an original requestor's signature
  to the serving DSA.

\m DUAs and DSAs which are able to provide the forward certification
  path when offering strong credentials.
\ei
The following DSA functionality will be provided optionally:
\bi
\m DSAs which strongly authenticate each other during DSABind when performing
  chained operations via DSP.

\m DSAs which digitally sign chained Directory operations when communicating
  to other DSAs via DSP.
\ei
The reason for making this provision optional is both that they are regarded 
to be less essential, and that their realization would require
enormous effort and would make interworking with the non-secure X.500 world
very complicated.

X.500 components with the above capabilities are regarded as secure X.500 components.
X.500 components which do not provide the above capabilities are regarded as
non-secure X.500 components. In particular, X.500 components which are only able
to perform simple authentication methods or less are regarded unsecure.


\subsection{Source Integrity and Source Confidentiality}

Signed Directory operations as described above only provide integrity protection
of the Directory information for the purpose of their communication via
DSP and DAP. They do not provide any integrity protection with respect to
the source of the information. The storage of signed information in the 
Directory would be an appropriate measure to gain more security to the user.

The only attribute types, however, which are stored in the Directory in signed 
form are Certificates and Certificate Revocation Lists., i.e. their source
integrity can be verified when receiving them from the Directory. The use of
other signed Directory information is out of the scope of X.500. However, it can
be regarded as an important security issue which must be solved in the future.

The situation is similar when considering confidentiality issues with Directories.
The storage of encrypted information in the Directory is out of the scope of
X.500. The encryption of Directory information for the purpose of their
communication via DAP and DSP is out of the scope of X.500, too.

\subsection{Local Protection of the DIB and the DSA Resources in a particular DSA}

Beside remote Directory operations via DAP and DSP, a particular DSA
and its part of the DIB is subject to manipulation through site personell
like DSA managers. There is a considerable potential threat through
possible insider attacks. In addition, since DSAs are connected to
public networks, there is a potential threat through possible communication
channels to the DSA other than those controlled by Directory protocols.
For instance, if an attacker is able to remotely log into a DSA via rlogin,
X.29, or ftp, and gains access to DSA resources or the information base,
the effect could be disastrous.

Threats of this kind are out of the scope of X.500. Local protection has to
be employed to counter this kind of threat.
