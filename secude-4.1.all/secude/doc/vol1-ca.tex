\section{The Duties of a Certification Authority (CA)}
\markboth{Duties of a CA}{Duties of a CA}
\thispagestyle{myheadings}
\label{ca}

\subsection{Certification Authorities}
\label{ca-dfn}

User-CAs sign user-certificates with their own private CA-keys.
Their keys are certified by other (higher) CAs.
This way CAs are structured within a certification tree hierarchy.
On top of it there is a so called ``Root CA'', the public key of which
is publicly well known and not certified.

Within SecuDe
a simple certification tree is proposed.
On top of the tree there is one ``Root-CA'', e.g.:
\begin{center}
C=DE;O=dfn;OU=CA.
\end{center}
Its function is for example to certify other organization-wide CAs:
\begin{center}
C=DE;O=uni-xxx;OU=CA.
\end{center}
Those organization-wide CAs could be user-CAs and issue certificates
to users within the organisation.

%\begin{figure}
\begin{center}
\makebox[6.500in][l]{
  \vbox to 1.847in{
    \vfill
    \special{psfile=vol1-fig5.ps}
  }
  \vspace{-\baselineskip}
}
\end{center}
\label{fig-ca-dfn}
\stepcounter{Abb}
{\footnotesize Fig.\arabic{Abb}: Certification tree}
%\end{figure}
\\ [1em]

The ``Root CA'' can certify user-certificates as well.
It also cross-certifies to other CAs outside the own tree.
One important cross-certification is to be established to the CA of the
US-Internet
\cite{rfc1}:
\begin{center}
C=US;O=RSA Data Security Inc.;OU=Internet Certifying Authority.
\end{center}
Other CAs below the Root CA can also cross-certify
inside and outside the own tree.

\subsection{The Duties of a user-CA}
\label{ca-uca}

A CA is said to be a user's user-CA if the user owns a certificate
issued by this CA.
User-CAs have the duty to supply their users with
{\em all information}
which they need in order
to use the signature and encryption services.
This includes

\begin{itemize}
\item
sending name and new public keys of the ``Root-CA''
in form of a trustworthy cross-certificate
whenever a ``Root-CA'' changes (or changes keys),
\item
sending the forward certification path whenever any higher
CA changes (or changes keys),
\item
maintenance of black lists of certificates
which this CA had issued and revoked,
\item
maintenance of its directory entries,
\item
updating the user-certificate attributes in the directory entries of its
users,
\item
issuing user-certificates.
\end{itemize}

It is {\em not} the user-CA to generate keys on behalf of their users.
Users generate their own keys.
Therefore, a user-CA cannot guarantee the quality or the uniqueness
of user-keys. The significance of a user-certificate is this:

\begin{itemize}
\item
the issuing CA knows that the certificate's owner uses this public key,
\item
the issuing CA proves the certificate's owner to be authentic,
\item
the issuing CA supplies the certificate's owner with the
most actual key information about the certification tree.
\end{itemize}

\subsection{Creation of User-Keys}
\label{ca-cuk}

Enrollment to the
signature and encryption services is a decentral act between a user
and any CA of his choice.
The user calls local functions in order to generate his
``software-smartcard''.
{\em The user generates his keys locally and sends a prototype
certificate to his user-CA. After checking the prototype
certificate, the user-CA will resend the signed certificate to the user}.
The user-CA will also send all certificates and public keys
of all CA's in the forward certification path of the user,
with which the user will complete the initialization of his
software-smartcard.

This way of decentralized key generation has three advantages over the
key generation at central authority stations.

First, the whole process is decentralized and releases
the central authorities from work. Users would go through
the key generation process only very rarely.
The argument, that the quality of keys cannot be {\em guaranteed
by the central authorities}, is correct.
However, the quality of the keys is in the interest of the users,
they would hurt nobody but themselves if they play wrong
and generate wrong or bad keys.
The quality of keys could be guaranteed by the key generation
software which might be well delivered by central authorities.

Second, private keys, which are highly sensitive information,
are never transferred through the network.

Third, this way it is guaranteed that private keys do exist only once,
namely in the hand of the users (i.e. in their smartcards).
This is different with the key management in the
``Privacy Enhanced Mail'' of the US-Internet \cite{rfc1},
where local servers keep the key information on behalf of
local users.
This is also different with the
Kerberos modell \cite{ste1},
where symmetric encryption is used
and where trusted authentication servers keep the symmetric keys
on behalf of all users.

The following procedure is taken from the
specifications of the ``Privacy Enhanced Mail''
of the US-Internet, Part IV \cite{rfc1}.
In order to get a user-certificate from a CA,
a user sends a {\em prototype certificate}
containing all information which the user already knows
and leaving all fields open which the user does not know:

\begin{itemize}
\item
{\em version}: default 0.
\item
{\em serial number}: ``1'' (will be replaced by CA).
\item
{\em name of issuing CA}.
\item
{\em validity}: chosen by user, can be restricted by CA.
\item
{\em subject}: user's name.
\item
{\em subjectPublicKeyInfo}: algorithm identifier, parameter and
bitstring of the public key generated by the user.
\item
{\em signature}: later, on the real certificate,
the signature must be composed by the CA.
Now, on the prototype certificate,
the user applies the private key which he had just generated
on the fields to be signed, and inserts the result, instead.
\end{itemize}

The prototype certificate is sent electronically to the CA of the user's
choice. The user also sends a printed version of the prototype signature field
by traditional paper mail
to the CA in order to enable the CA to check the user's authentic
identity.
The CA will not be able to check the quality of the user's key.
However, she will be able to check the integrity of the certificate fields.
The CA will also be able to recognize
that the user's public key is indeed paired with a private
key in the hand of the user.
The CA replaces the serial number and the validity time frame,
signs the certificate with its private CA-key,
and sends the certificate electronically back to the user.
The CA will also send its electronic signature per paper mail
in order to prove the certificate authentic.
This happens with both certificates: the public verification key
certificate and the public encryption key certificate.
The CA includes the following information:

\begin{itemize}
\item both user certificates,
\item the forward certification path (``FCPath''),
\item the public keys of the ``Root-CA'' (``PKRoot''),
\item a list of public verification keys of all CAs in the
forward certification path (an initial value of ``PKList''),
\item a list of public encryption keys of all CAs in the
forward certification path (an initial value of ``EKList'').
\end{itemize}

Once a user is assigned to a CA he can change keys whenever he feels like.
This happens the same way as described above.
In order to extend a certificate's validation time the user will
also generate a prototype certificate. It is up to the user
if he changes keys in line with a new certificate.
