\section{Operation of Encryption Keys}
\markboth{Operation Encryption}{Operation Encryption}
\thispagestyle{myheadings}
\label{ope}

\subsection{One Extra Pair of Keys for Encryption and Decryption}
\label{ope-epk}

Decryption of an encrypted text means to apply one's private key
to an unreadable text. In order to prevent users from ``signing''
unreadable texts blindly,
users are provided by {\em two different pairs of keys}:
one pair for signature and verification; the other pair for
encryption and decryption.

The public encryption key is certified by the same authority which
issues the signature certificate to the user. Therefore,
the encryption user certificate is combined with the same
forward certification path to make up an
{\em encryption recipient certificate}
(X.509 type ``certificates'').

A signature originator certificate is distinguished from
an encryption recipient certificate by the object identifiers
of their public key infos.
The standards (X.509, Annex H, \cite{cci4}) define
object identifiers for {\em signature algorithms}
and for {\em encryption algorithms}, respectively.
See paragraph \ref{ds-ca} above for reference.

\subsection{Security Requirements}
\label{ope-sr}

The application of a wrong encryption key is not that dangerous,
in that the main effect will be, that nobody in the world is able
to decrypt the text. In that case, the recipient might provide
the sender with the correct public encryption key.
However, one must be aware of intruders
who try to trigger information by offering masqueraded
encryption recipient certificates.
This is encountered by certificate checks:
before encrypting a text for a partner,
a user checks the encryption recipient certificate at an
appropriate security level.

\subsection{Change of Keys}
\label{ope-ck}

\subsubsection{Old and New Decryption Key}
\label{ope-ondk}

In order to avoid confusion by a change of encryption keys,
every user has not only one, but {\em two}
private decryption keys on his smartcard
(DecryptionSKnew and DecryptionSKold are described
in paragraph \ref{decsk} above):

\begin {center}
\begin {tabular}{|c|c|c|}
\hline
{\em issuer} & {\em serial no}& {\em key} \\ \hline
$<issuer>_{Cert-old}$ & $<serial-no>_{Cert-old}$ & $B_{Sold}$    \\ \hline
$<issuer>_{Cert-new}$ & $<serial-no>_{Cert-new}$ & $B_{Snew}$    \\ \hline
\end {tabular}
\end {center}
 
\label{fig-ope-ondk}
\stepcounter{Abb}
{\footnotesize Fig.\arabic{Abb}: Private decryption keys}
\\
{\footnotesize $<issuer>_{Cert-old}$ denotes the issuer name of the
old decryption certificate;
$<serial-no>_{Cert-old}$ denotes the serial number of the
old decryption certificate;
$B_S$ denotes the private (``secret'') key
of the user $B$, in this case for the purpose of decryption}
\\[1em]
A sender of an encrypted message should inform the recipient,
which public encryption key he used,
in order to enable the recipient to apply his correct private decryption key.
There is only one unique way to do that without using redundant
message encoding.
That is, by sending the issuer name and the serial number
of the public encryption certificate along with the encrypted message.
The recipient has to map that information on his respective private
decryption key.
Therefore, the smartcard table for the old and
new private decryption key is organized this way.
Also, this allows the user to store more than two decryption keys,
and even cross-certificate mappings for the keys.
\\
In case of confusion, the recipient can ask the sender
for the encryption recipient certificate in order to check it;
he can (and should) also provide the sender by the correct certificate.

\subsubsection{Reencrypting Stored Encrypted Texts}
\label{ope-set}

When receiving a new pair of decryption keys and a new
encryption recipient certificate,
the user has to decrypt all of his stored encrypted texts
by applying the old decryption key,
and then to encrypt them again
by applying the new encryption key, which he reads from the new certificate.
In general, this applies not for the complete texts, but only for
the DES-keys, by which the texts are symmetrically encrypted.
The DES-keys will not change, such that the texts themselves
need not be reencrypted.

\subsubsection{Steps to Change Keys}
\label{ope-sck}

User keys are created by the users themselves.
In case of a change of a pair of encryption and decryption keys,
on the smartcard
the expired secret decryption key will switch its status from
``new'' to ``old'', while the newly created secret decryption
key will be stored with the status ``new''.
The public encryption key would be included in a prototype
certificate which is sent to the user-CA with request of
updating and signing the certificate.
After reception of the signed certificate the user
replaces the old encryption key certificate on his smartcard
by this new one. The old one is not needed any more
(``EncrCertificate'', see \ref{enccert}).
Now, the user sends his new encryption certificate
to all his important communication parnters.
He also reencrypts all his locally stored encrypted texts (see \ref{ope-set}).

\subsection{The Smartcard Table
``Trusted Public Encryption Keys''}
\label{ope-tpvk}

In order to support confidential communication between partners without
use of directories and without repeated checks of certificates,
users maintain tables of trusted public encryption keys
on their smartcards.
The users can read and write the lines of their smartcard tables.

Before adding a new public key to that list, the respective
encryption recipient certificate is to be checked at security level 3.
After that, the certificate is thrown away and will not be used any more.

There is no security risk with the late update of such an entry.
When using an old encryption key, the partner is able to decrypt the
text anyway, in that he keeps one old decryption key on his smartcard.
If the encryption key applied is even older, the recipient will not
be able to decrypt the text, and no person in the world would be able,
either. This is the occasion to provide the partner with the actual
encryption key.
