% list.tex - minimal implementation of single level lists

% \list{(\letters)}
%      \item Text of first item. More text 
%	     of first item.
%      \item Text of second item.
%      \item Text of third item.
%      \endlist

% will produce

%    (a) Text of first item.  More text
%        of first item.
%
%    (b) Text of second item.
%
%    (c) Text of third item.

% Using the functions

\def\list	% which begins a list
#1% form of the item markers
{\par\begingroup \itemnumber=0 \parindent=0pt \parskip=4pt \curmark={#1}}%

\def\endlist{\par\endgroup}	% which ends a list

% and 

\def\item{\par\indent \advance\itemnumber by 1
	  \hangindent=1em \hangafter=0 
	  \llap{\the\curmark\enspace}\ignorespaces}

% which marks the end of one item and the beginning of the next.

% The list mechanism requires one count to keep track of the
% item number and one token for the item marker.

	\newcount\itemnumber 
	\newtoks\curmark

% It also requires a variety of functions that indicate in a
% convenient way the form of the item markers and several are 
% given below.  These can be used alone or in combination with
% constant text to make a variety of forms.

\def\bullets{$\bullet$}
\def\dashes{{\rm ---}}
\def\dots{$\cdot$}
\def\letters{\alph{\the\itemnumber}}
\def\Letters{\Alph{\the\itemnumber}}
\def\numbers{\the\itemnumber}
\def\roman{\expandafter\romannumeral\the\itemnumber}
\def\Roman{\uppercase\expandafter{\romannumeral\the\itemnumber}}

