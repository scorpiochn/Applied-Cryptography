\section{Implementation Strategy}
\markboth{Implementation Strategy}{Implementation Strategy}
\thispagestyle{myheadings}
\label{isw}

\subsection{Summary}
\label{isw-is}

A general strategy is to develop different modules,
which are interrelated by a services provider -- services consumer concept,
and which offer programming interfaces to interested programmers,
who want to embed the modules into other, possibly more complex,
application contexts.
The following modules are about to be realized:

\begin{enumerate}
\item
{\em Secure storage and cryptography}:
A cryptographic system for RSA-, DSA- and DES-algorithms
as well as hash functions,
connected to a Personal Security Environment (PSE).
It provides functions which perform
cryptographic algorithms and a secure storage of keys and other
personal local information.
The smartcard technology
is not visible at the interface. The smartcard service realized
so far, is the so called ``software smartcard''.
\item
{\em Authentication framework certificate handling}:
A richer functionality
of cryptographic functions and secure storage of keys
according to X.509 equipped with a more complex structure
due to the description of these principles of security operations.
There are two submodules, one supporting
{\em local} (i.e. smartcard oriented)
handling of signatures and certificates;
the other one supports {\em distributed} (i.e. directory based)
certificate and black lists information.
\item
{\em ``Pricacy Enhanced Mail''} (``PEM'')
functions which {\em protect messages} according to the
Internet specifications RFC 1421 - 1423 \cite{rfc1}.
The protected messages can be securely
exchanged within an unprotected network.
\item
{\em Key management functions}:
There are functions to generate encryption keys and signatures,
other functions to support secure interchange of security information
like certificates, pairs of root-CA keys, etc.
These functions are of interest for certification authorities,
and for users to maintain their PSEs.
\item
{\em Auxiliary functions}
of general applicability to support all modules.
These functions include octetstring handling, file handling,
printable representation of SecuDE objects,
support of ASN.1 encoding/decoding, etc.
\end{enumerate}

\subsection{The Modular Structure}
\label{isw-sm}

The specifications describe basic functions in terms of C-language
procedure calls. They constitute a basic set of software modules
which are intended to support {\em security applications},
in that they provide {\em security application programming interfaces}.
The applications {\em Secured X.500} and {\em PEM} (Privacy Enhanced Mail, RFC 1421-1424)
are based on these functions. They comprise the
application programming interfaces {\em Secure-IF},
{\em AF-IF}, and {\em PEM-IF} as shown in Fig. 17 below.
This interface structure was chosen for the following reasons: \\ [1ex]
Security applications should be independent of underlying technology, i.e. it should
not make any difference for the security functionality in an application whether 
cryptographic algorithms are realized
in software or hardware chips or in external devices like smartcards or smartcard readers,
nor should it make a difference how and where security relevant information like keys 
are stored. They need an interface which largely hides the implications of particular
technology and which is oriented on user required security characteristics
in order to support migrations to advanced technologies. For this purpose
we defined the interface {\em Secure-IF}.
Secure-IF provides {\em Cryptographic Functions} 
and a {\em Personal Security Environment} (PSE) in a technology independent form. \\ [1ex]
The {\em Personal Security Environment} can be realized by various methods. The highest level of
security against unallowed manipulation of security relevant information requires
hardware support. By current standards, {\em smartcards} are appropriate means to
store security relevant personal data because they combine the two aspects
\bi
\m security against eavesdropping through hardware properties and
\m mobility and portability of personal information.
\ei
However, smartcard technology
is currently developing fast, and standardisation in this area is not stable
yet. For this reason, we provide a software substitute for a smartcard environment. 
We assume that all security relevant functions are
performed by software (e.g. in work stations, PC's or main frames) and that all
security relevant information is stored on disks or other background memory
of the systems. Protection mechanisms as they are offered by smartcards are to
be modelled by means of file encryption and electronic signatures. Therefore
we speak of the so called {\em Software PSE} (SW-PSE) to express the analogy of 
security functionality of the smartcard and its software substitute. 

\begin{figure}
\framebox[3.1cm][c]{\rule[-0.5cm]{0cm}{0.9cm}X.500}
\hspace*{0.39cm}
\framebox[3.1cm][c]{\rule[-0.5cm]{0cm}{0.9cm}PEM}
\hspace*{0.4cm}
\framebox[2.9cm][c]{\rule[-0.5cm]{0cm}{0.9cm}Key Managm.}
\hspace*{0.4cm}
\framebox[3.1cm][c]{\rule[-0.5cm]{0cm}{0.9cm}Other Applic.}
\vs{0.3cm}
{\bf
$|$ -- X500-IF -- -- $|$ \hspace*{0.2cm} $|$ -- PEM-IF -- -- $|$ \hspace*{0.3cm} $|$ -- KM-IF -- -- $|$
}
\vs{0.3cm}
\framebox[3.1cm][c]{\rule[-0.5cm]{0cm}{0.9cm}X.500 Support}
\hspace*{0.39cm}
\framebox[3.1cm][c]{\rule[-0.5cm]{0cm}{0.9cm}PEM Support}
\hspace*{0.4cm}
\framebox[2.9cm][c]{\rule[-0.5cm]{0cm}{0.9cm}KM Support}
\vs{0.3cm}
{\bf
$|$ -- -- -- -- -- -- -- -- -- -- -- -- -- -- -- -- -- AF-IF -- -- -- -- -- -- -- -- -- -- -- -- -- -- -- -- $|$
%$|$ -- -- -- -- -- -- -- -- -- -- -- -- AF-IF -- -- -- -- -- -- -- -- -- -- -- $|$
}
\vs{0.3cm}
\framebox[5.5cm][c]{\rule[-0.5cm]{0cm}{0.9cm}Local Certificate Handling}
\hspace*{0.4cm}
\framebox[4.10cm][c]{\rule[-0.5cm]{0cm}{0.9cm}Directory Access}
\hspace*{0.4cm}
\framebox[3.1cm][c]{\rule[-0.5cm]{0cm}{0.9cm}Auxil. Functions}
\vs{0.3cm}
{\bf
$|$ -- -- -- -- -- -- -- -- -- -- -- -- -- -- -- -- Secure-IF -- -- -- -- -- -- -- -- -- -- -- -- -- -- -- $|$
}
\vs{0.3cm}
\framebox[13.8cm][c]{\rule[-0.5cm]{0cm}{0.9cm}Personal Security Environment and Cryptography (technology dependant)}
\vs{0.3cm}
{\bf 
$|$ -- -- -- SCA-IF -- -- -- $|$
}
\vs{0.3cm}
\framebox[4.2cm][c]{\rule[-0.5cm]{0cm}{0.9cm}SC-Environment}
\hspace*{4.05cm}
\framebox[2.5cm][c]{\rule[-0.5cm]{0cm}{0.9cm}Crypto-SW}
\hspace*{0.2cm}
\framebox[2.5cm][c]{\rule[-0.5cm]{0cm}{0.9cm}SW-PSE}
%\vs{0.5cm}
%\caption{Interface Structure}
%\label{bild1}
\\[1em]
\stepcounter{Abb}

{\footnotesize Fig.\arabic{Abb}:
Software Modules Interface Structure}
\label{ifstruct}
\end{figure}

Security of the applications X.500 and PEM is based on
certification procedures and formats as defined in
{\em Volume 2: Specification of Security Interfaces}
of SecuDe in accordance to {\em CCITT X.500ff}.
An essential part of it
is the {\em Authentication Framework} X.509. For this functionality we defined the interface
{\em AF-IF} (AF stands for Authentication Framework).
One AF module provides the local handling of certificates
and a certification path oriented signature verification procedure.
Another AF module provides access to directories in order to
support distributed security information.
Directory access will particularly be used to retrieve or store
certificates and black lists,
both by certification authorities and ordinary users.
It can be expected, however, that
the functionality defined by {\em AF-IF} is not limited to the purpose of Electronic Mail, but that
it is useful for a wide range of applications. On the other hand, one can imagine other
applications, e.g. using ECMA certification methods, which use {\em Secure-IF} directly.
\\[1em]
The interface {\em PEM-IF} was defined for the purpose of the application {\em Privacy Enhanced 
Mail}. It comprises functions which are necessary in the context of RFC 1421 - 1423.
\\[1em]
Naturally, a security system based on asymmetric cryptography
depends on a well organized {\em key management} (``KM'') infrastructure
including certification authorities, users and directories.
For this purpose, the KM module is defined, comprising
functions to generate keys and certificates,
and other functions to support the secure exchange of security information
such as keys and certificates.
\\[1em]
Additionally, there is
a collection of useful auxiliary functions mostly destined for the
handling of octetstrings, object identifiers, algorithm identifiers, and PEM-specific
transformations. Some of them are of internal use only, and the application programmer will
not need to handle them, but they are mentioned here for completeness.

\subsection{Secure Storage and Cryptography}
\label{isw-sec}

This module consists of two submodules.
One submodule performs cryptographic algorithms.
The other submodule performs secure storage of data on a local smartcard.
The cryptographic submodule uses the service of the secure storage submodule,
in that cryptographic keys on the smartcard can be used.

The cryptographic submodule contains the following set of functions:
There is a set of functions which performs arithmetic operations
on very large integer numbers. These functions are available in C
and in a number of assembler codes for different CPU platforms
for efficiency reasons.
There are asymmetric RSA and DSA functions based on these arithmetic functions.
There are symmetric DES
encryption and decryption functions.
There are also functions to perform different hash algorithms,
which are used in the asymmetric signature and verification functions.
There are functions to generate keys and keep them
in a local reference table or store them in a PSE.
There is also a function to generate a DES-key
and encrypt it with an asymmetric public key,
and the inverse hereof.
Finally, the generation of a random number is also supported
by this submodule.

The other submodule provides functions with the help of which
sensitive data (such as keys or certificates)
can be stored on or retrieved from a PSE.
This submodule comprises functions that simply
open, close, create, read or write a named PSE object.
The handling of pin-protection for the PSE and for
objects on the PSE is performed by this
submodule.

\subsection{Authentication Framework: Certificate Handling}
\label{isw-af}

This module supports the handling of X.509 data formats and procedures.
It consists of two submodules. One of the submodules
deals with the local handling of keys and certificates,
in that it accesses the PSE.
The other submodule deals with the remote security information,
in that it accesses an X.500 directory.

The local-AF submodule
invokes the functions for secure storage and cryptography
but embeds them into a richer structure due to X.509.
For example, this module defines and utilizes the certificates
structure in order to verify a signature.
It also defines and utilizes the X.509 format of a certificate
or a Key Information when accessing PSE objects.
This submodule ``knows'' the structure of the PSE objects.
It is by purpose, that the design of this AF programming interface
is hierarchically ``above'' the SEC interface.
This is in order to give one complete programming interface to programmers
who want to implement an X.509 oriented security application.

The remote-AF submodule
deals with the same X.509 formats.
However, it supports the application programmer
to implement programs for users
to retrieve information
from or store information into a distributed directory data base.
This is required by certification authorities in order to maintain
their certificates and black lists of certificates in directory attributes.
This is also required by users who maintain their certificates in
directory attributes.
Beside these security information, this submodule provides also access to
directory name attributes in order to support users and CAs
in their secure certificate and message handling.

\subsection{Key Management}
\label{isw-km}

This module supports users and certification authorities
to maintain consistent security information
within the communication environment.
The communication community needs support
to keep current and correct
information on PSEs, in directory entries,
and to exchange security information.
The principles of key operations are described in previous chapters
of this volume.
The following functions are provided:

\begin{itemize}
\item create a PSE
with the objects defined in the authentication framework module
and with initial contents,
\item check PSE for validity and consistency,
\item display PSE objects,
\item create a certificate or prototype certificate,
\item send a a certificate or prototype certificate,
\item extract oneself's certificate
from a message and store it on the PSE,
\item extract a partner's certificate
from a message and store it in a local certificate list,
\item edit PSE certificate lists,
\item check certificate format and content from a CA's point of view
(including uniqeness of public key),
\item sign certificate,
\item check certificate format and content from the owner's point of view,
\item send a a certification path,
\item extract a certification path
from a message and store it in the PSE,
\item send a pair of root-CA keys,
\item extract a pair of root-CA keys
from a message and store it in the PSE,
\item maintain directory entry ``Old Certificates'',
\item change signature keys (for users, CAs, root-CAs),
\item change encryption keys,
\item inform others about change of keys (for users, CAs, root-CAs),
\item maintain black lists of certificates.
\end{itemize}
