\section{ASN.1 Definitions of Object Structures}
\label{asn1}
\markboth{ASN.1 Definitions}{ASN.1 Definitions}
\thispagestyle{myheadings}

Both information which is exchanged via communication protocols,
and data which are stored locally in PSE objects 
(PSE:Personal Secure Environment, like smartcards),
are formatted according to the
Basic Encoding Rules of ASN.1.
During procession, however,
they are stored in C-structures
defined by the {\em Secure-IF}, the {\em AF-IF} and the 
{\em X400-IF} specifications.
Encoding and decoding between the ASN.1-structures
of the protocol elements or the PSE objects 
and the internal C-structures
are performed by {\em e\_*} and {\em d\_*} functions
provided at the {\em Secure-IF}, the {\em AF-IF} 
and the {\em X400-IF} interface.
In this section the ASN.1 structures
of the protocol and storage objects are defined.
The encoding- and decoding functions are mentioned, too.

\subsection{Object Identifiers}
\label{asn1-oid}

The specification of the following object identifiers
is according to the ISO standards 8824 and 6523.
ISO 8824-1987: Specification of Abstract Syntax Notation One (ASN.1),
ISO 6523-1984: Data interchange --
Structure for the identification of organizations. The following specifications 
are made for the DFN (Deutsches ForschungsNetz).
Since DFN is not an ISO identified organisation
we provisionally adopt an organizational code for {\em DFN}
due to ISO 6523, paragraph 5.2, in that an unassigned
international designator code ``9997'' for an
organisation ``44'' under ISO is assumed.
If DFN becomes an ISO identified organization, or if 
DFN security policies and DFN security procedures will be
embedded into a larger international context, 
both the designator code and the organization code for DFN
shall change.

{\small
\begin {center}
\begin {tabular}{lll}
ICD & 9997 & -- -- not yet assigned \\
Organization Code & dfn(44) & -- -- not definitive \\
Organization Name & DFN & \\
 & & \\
\end {tabular}
\end {center}
}

With these intermediate definitions
the object identifiers used within SecuDe
can be specified.

{\small
\begin {center}
\begin {tabular}{lll}
id-dfn ::= &
\multicolumn{2}{l}{\{ iso(1) identified-organization(3) icd(9997) dfn(44) \}
-- -- organization} \\
 & & \\
id-dfn-specs ::= & \{ id-dfn 0 \} & -- -- specifications \\
id-dfn-mods ::=  & \{ id-dfn 1 \} & -- -- modules \\
id-dfn-sp ::= & \{ id-dfn 2 \} & -- -- security policies \\
id-dfn-aid ::= & \{ id-dfn 3 \} & -- -- algorithm identifiers \\
id-dfn-attr ::= & \{ id-dfn 4 \} & -- -- attribute types \\
 & & \\
id-dfn-sosi ::= & \{ id-dfn-specs 0 \} & -- specification of security interfaces \\
 & & \\
id-dfn-secc ::= & \{ id-dfn-mods 0 \} & -- secure cryptography module \\
id-dfn-secs ::= & \{ id-dfn-mods 1 \} & -- secure storage module \\
id-dfn-afl ::= & \{ id-dfn-mods 2 \} & -- local authentication framework module \\
id-dfn-afd ::= & \{ id-dfn-mods 3 \} & -- directory authentication framework module \\
id-dfn-pem ::= & \{ id-dfn-mods 4 \} & -- privacy enhanced mail module \\
id-dfn-x400 ::= & \{ id-dfn-mods 5 \} & -- secure x.400 module \\
id-dfn-x500 ::= & \{ id-dfn-mods 6 \} & -- secure x.500 module \\
id-dfn-km ::= & \{ id-dfn-mods 7 \} & -- key management module \\
 & & \\
id-dfn-ssp ::= & \{ id-dfn-sp 0 \} & -- simple security policy for mail \\
\end {tabular}
\end {center}
}

The object identifiers for algorithm identifiers are defined
in X.509, Annex H.
The object identifier for a message token and an asymmetric token
are defined in X.411, Annex A.
{\small
\begin {center}
\begin {tabular}{lll}
algorithm ::= & \{ 2 5 8 \} & -- -- from X.501, Annex B \\
encryptionAlgorithm ::= & \{ algorithm 1 \} & ::= \{ 2 5 8 1 \} \\
hashAlgorithm ::= & \{ algorithm 2 \} & ::= \{ 2 5 8 2 \} \\
signatureAlgorithm ::= & \{ algorithm 3 \} & ::= \{ 2 5 8 3 \} \\
 & & \\
id-tok ::= & \{ 2 6 3 6 \} & \\
id-tok-asymmetricToken ::= & \{id-tok 0 \} & ::= \{ 2 6 3 6 0 \} \\
\end {tabular}
\end {center}
}

The object identifiers of attribute types are defined
in X.520.
{\small
\begin {center}
\begin {tabular}{lll}
attributeType ::=          & \{ 2 5 4 \} & -- -- from X.501, Annex B \\
commonName ::=             & \{ attributeType 3 \}  & ::= \{ 2 5 4 3 \}  \\
surname ::=                & \{ attributeType 4 \}  & ::= \{ 2 5 4 4 \}  \\
countryName ::=            & \{ attributeType 6 \}  & ::= \{ 2 5 4 6 \}  \\
organizationName ::=       & \{ attributeType 10 \} & ::= \{ 2 5 4 10 \} \\
organizationalUnitName ::= & \{ attributeType 11 \} & ::= \{ 2 5 4 11 \} \\
localityName ::=           & \{ attributeType 7 \}  & ::= \{ 2 5 4 7 \}  \\
stateOrProvinceName ::=    & \{ attributeType 8 \}  & ::= \{ 2 5 4 8 \}  \\
streetAddress ::=          & \{ attributeType 9 \}  & ::= \{ 2 5 4 9 \}  \\
title ::=                  & \{ attributeType 12 \} & ::= \{ 2 5 4 12 \}  \\
serialNumber ::=           & \{ attributeType 5 \}  & ::= \{ 2 5 4 5 \}  \\
businessCategory ::=       & \{ attributeType 15 \} & ::= \{ 2 5 4 15 \}  \\
description      ::=       & \{ attributeType 13 \} & ::= \{ 2 5 4 13 \}  \\
userCertificate ::=        & \{ attributeType 36 \} & ::= \{ 2 5 4 36 \} \\
cACertificate ::=          & \{ attributeType 37 \} & ::= \{ 2 5 4 37 \} \\
authorityCrl ::=   & \{ attributeType 38 \} & ::= \{ 2 5 4 38 \} \\
certificateCrl ::= & \{ attributeType 39 \} & ::= \{ 2 5 4 39 \} \\
revokedCertificateList ::= & \{ id-dfn-attr 0 \} & ::= \{ 1 3 9997 44 4 0 \} \\
oldCertificateList ::=     & \{ id-dfn-attr 1 \} & ::= \{ 1 3 9997 44 4 1 \}
\end {tabular}
\end {center}
}

\subsection{Algorithm Identifiers}
\label{asn1-algid}
 
The format of an algorithm identifier
is defined in the Directory Authentication Framework
X.509, Annex H.
It consists of an object identifier and a parameter,
which is of any type, defined by the object identifier.
Object identifier definitions are taken from X.509,
NIST OIW, EWOS, and TeleTrusT Germany.
See paragraph \ref{algid} for the algorithm identifiers
and their default parameters defined within SecuDE.
\\ [1em]
Encoding/Decoding: e\_AlgId(), d\_AlgId().

\subsection{Name Format}
\label{asn1-name}
 
The X.500 recommendations on the directory specify
names for the purpose of unique identification of the directory objects
(X.501, Directory Information Model).
For example, names are assigned to human users.
Also, issuer and subject of a certificate are identified by their names.
The name format is defined in
X.501, Annex C.
{\small
\begin {center}
\begin {tabular}{ll}
DistinguishedName & ::= RDNSequence              \\
RDNSequence  & ::= {\bf SEQUENCE OF} RelativeDistinguishedName \\
 & \\
RelativeDistinguishedName & ::= {\bf SET OF} AttributeValueAssertion \\
 & \\
AttributeValueAssertion   &
   ::= {\bf SEQUENCE} \{ AttributeType, AttributeValue \} \\
 & \\
AttributeType  &  ::= {\bf OBJECT IDENTIFIER} \\
AttributeValue &  ::= {\bf ANY} -- -- defined by AttributeType
\end {tabular}
\end {center}
}
The directory attribute {\em DistinguishedName}
is a sequence of {\em RelativeDistinguishedName}
which in turn is a set of attributeType/attributeValue pairs.
Within SecuDe
only a limited set of attribute types is used to construct relative distinguished names,
these are:
countryName, organizationalName, organizationalUnitName, commonName, surname,
title, serialNumber, localityName, streetAddress, and businessCategory.
All other attribute types would be ignored.
\\ [1ex]
SecuDE handles two representations of distinguished names. One is the {\em DName}
data type which is a C-structure corresponding to the ASN.1 notation below. 
{\small
\bvtab
\2 typedef struct \{                            \\
\4        RDName       \3 *element\_IF\_2;      \\
\4        DName        \3 *next;                \\
\2 \} DName;                                    \\ [1em]
\2 typedef struct \{                            \\
\4        AVA          \3 *member\_IF\_0;       \\
\4        RDName       \3 *next;                \\
\2 \} RDName;                                   \\ [1em]
\2 typedef struct \{                            \\ 
\4        ObjId  \3 *element\_IF\_0;      \\
\4        PElement     \3 *element\_IF\_1;      \\
\2 \} AVA;                                      \\
\evtab
}
The second which is used at the SecuDE utility interface is the {\em Name} data type.
This is a printable character string representation
of a distinguished name in the form
\bc
{\small keyword=value \% keyword=value ... ; keyword=value \% keyword = value ...; ...}
\ec
keyword is a name for an attribute type. 
The ';' delimiter separates the RDN members of the DN, i. e. each 
keyword=value \% keyword=value ... expression denotes one relative distinguished name.
The '\%' delimiter separates the attributes within one RDN.
The following keywords, representing X.520 defined attributes,  are possible:
{\small
\bvtab
C   \2 countryName \\
O   \2 organizationName  \\
OU  \2 organizationalUnitName \\
S   \2 surname \\
CN  \2 commonName \\
L   \2 localityName \\
SP  \2 stateOrProvinceName \\
ST  \2 streetAddress \\
T   \2 title \\
SN  \2 serialNumber \\
BC  \2 businessCategory \\
D   \2 description 
\evtab
}
RDNs must be ordered top down. Letter cases are not significant for names, 
as well as spaces may be used inside and outside the names. \\ [1ex]
Example: \\
{\small
C=DE; O=gmd; OU=I2 \% CN=Institut fuer TeleKooperationsTechnik; CN=Wolfgang Schneider} 
\\ [1em]
Encoding/Decoding: e\_DName(), d\_DName(), 
which transform between the C-structure DName and the ASN.1 DER code.
There are other name transformation functions which transform between
the printable representation {\em Name} and the C-structure {\em DName}: \\
aux\_Name2DName(), aux\_DName2Name() (see AUX\_NAME(3X)). \\ [1ex]

\subsection{Some Directory Attributes}
\label{asn1-x500}

In X.501, Directory Information Model (7.2.3), a directory attribute
is defined to consist of an {\em attribute type},
which identifies the class of information given by an attribute,
and the corresponding {\em attribute values(s)}, which are the
particular instances of that class appearing in the entry

{\small
\begin {center}
\begin {tabular}{lll}
Attribute &
   ::= {\bf SEQUENCE} \{ AttributeType, SET OF AttributeValue \} \\
 & \\
AttributeType  &  ::= {\bf OBJECT IDENTIFIER} \\
AttributeValue &  ::= {\bf ANY} -- -- defined by AttributeType
\end {tabular}
\end {center}
}

In the following subsections, some attribute values are described.

\subsubsection{Certificate}
\label{asn1-cert}

{\small
\begin {center}
\begin {tabular}{lll}

Certificate ::= & \multicolumn{2}{l} {{\bf SIGNED SEQUENCE} \{ } \\
  & version [0]          & INTEGER DEFAULT 0,          \\
  & serialNumber         & INTEGER,                    \\
  & signature            & AlgorithmIdentifier,        \\
  & issuer               & Name,                       \\
  & validity             & Validity,                   \\
  & subject              & Name,                       \\
  & subjectPublicKeyInfo & SubjectPublicKeyInfo \}     \\
\end {tabular}
\end {center}
}
 
{\em Certificate} is defined by X.509, Annex G.
{\em Algorithm\-Identifier} is defined in X.509, Annex G,
as a SEQUENCE of an Object Identifier and a parameters type
which depends on the algorithm.
{\em Name} is defined in X.501, Annex C,
as a SEQUENCE of Relative Distinguished Names,
which are sets of Attribute-Value-Assertions.
{\em Validity} is defined in X.509, Annex G,
as a SEQUENCE of the UTCTimes {\em notBefore} and {\em notAfter}.
\\[1ex]
{\em SubjectPublicKeyInfo} is defined by X.509, Annex G.
The substructure of the public key BitString is defined by
X.509, Annex C, C.3.

{\small
\begin {center}
\begin {tabular}{lll}

SubjectPublicKeyInfo  ::= & {\bf SEQUENCE} \{ & \\
  & algorithm         & AlgorithmIdentifier,    \\
  & subjectPublicKey  & BIT STRING \}           \\
  & \multicolumn{2}{l} {-- --} \\
  & \multicolumn{2}{l} {-- -- the BitString of subjectPublicKey}    \\
  & \multicolumn{2}{l} {-- -- contains the ASN.1 encoding of a sequence of} \\
  & \multicolumn{2}{l} {-- -- the RSA-modulus n and the public RSA-exponent e:    }   \\
  & \multicolumn{2}{l} {-- --} \\
  & -- -- SEQUENCE \{ & n INTEGER,   \\
  & -- --             & e INTEGER \}
\end {tabular}
\end {center}
}

See also X.509, Annex G, for the SIGNED MACRO,
which is replaced by the following MACRO:

{\small
\begin {center}
\begin {tabular}{lll}
SIGNED MACRO   ::= & & \\
BEGIN              & & \\
TYPE NOTATION  ::= & type(ToBeSigned)     & \\
VALUE NOTATION ::= & type(VALUE           & \\
		   & SEQUENCE \{          & \\
		   & ToBeSigned,          & \\
		   & AlgorithmIdentifier, & \\
 & \multicolumn{2}{l} {-- -- of the algorithm used to generate}  \\
 & \multicolumn{2}{l} {-- -- the signature}                      \\
		   & BitString \}          & \\
 & \multicolumn{2}{l} {-- -- where the BIT STRING is the result} \\
 & \multicolumn{2}{l} {-- -- of the encryption of the value}     \\
 & \multicolumn{2}{l} {-- -- of the hash of ToBeSigned}          \\
 & ) & \\
END & &
\end {tabular}
\end {center}
}
 
Encoding/Decoding: e\_Certificate(), d\_Certificate().


\subsubsection{Black Lists}
\label{asn1-blists}

{\small
\begin {center}
\begin {tabular}{lll}
revokedCertificateList & \multicolumn{2}{l} {::= {\bf SIGNED SEQUENCE} \{ } \\
  & signature     & AlgorithmIdentifier,        \\
  & issuer        & Name,                       \\
  & list          & {\bf SEQUENCE OF} RCLEntry, \\
  & lastUpdate    & UTCTime,                    \\
  & nextUpdate    & UTCTime \}                  \\
  &               &                             \\
RCLEntry & \multicolumn{2}{l} {::= {\bf SEQUENCE} \{ }  \\
  & subject        & CertificateSerialNumber,   \\
  & revocationDate & UTCTime \}
\end {tabular}
\end {center}
}

The attributes ``CertificateRevocationList'' and ``AuthorityRevocationList''
are defined in X.509, Annex G. They are not used within SecuDe.
SecuDe uses the
{\em Revoked Certificate List} defined by
RFC 1114 instead.
\\[1ex]
Encoding/Decoding: e\_Crl(), d\_Crl().

\subsubsection{Old Certificates}
\label{asn1-oldcerts}

This attribute is not defined in X.500. Its purpose within SecuDE
is to support old signatures.

{\small
\begin {center}
\begin {tabular}{ll}
OldCertificateList & ::=  {\small\bf SEQUENCE OF SEQUENCE} \{        \\
		   & serialNumber     CertificateSerialNumber,       \\
		   & crossCertificate Certificate \}
\end {tabular}
\end {center}
}

Encoding/Decoding: e\_OCList(), d\_OCList().

\subsection{X.400 Security Extension Elements}
\label{asn1-x400}
 
Volume 1: Principles of Security Operations, of this project description
gives a comprehensive overview over the ASN.1 formats of the
X.400 extension elements. It also suggests e certain selection
of security extension elements and their subfields.
This paragraph summarizes the selected extension elements
and also omits the unused subfields.

This way, the following paragraphs define the simple DFN security policy
identified by the identifier id-dfn-ssp.

\subsubsection{Originator Certificate}
\label{asn1-origcert}

{\small
\btab
\1 originator-certificate ::= {\bf SEQUENCE} \{ \\
\2         type \3         [0] INTEGER, -- -- 15 \\
\2         criticality \3  [1] BitString, -- -- 2 for delivery \\
\2         value \3        [2] Certificates \} \\
\\
\etab
}

The protocol data units of both, Certificates and Certificate is defined in
X.509, Annex G.

\subsubsection{Security Label}
\label{asn1-seclab}

{\small
\btab
\1 message-security-label ::= {\bf SEQUENCE} \{ \\
\2         type \3         [0] INTEGER, -- -- 20 \\
\2         criticality \3  [1] BitString, -- -- 2 for delivery \\
\2         value \3        [2] SecurityLabel \} \\
\\
\1 SecurityLabel ::= {\bf SET} \{ \\
\2         security-policy-identifier \3 OBJECT IDENTIFIER, -- -- id-dfn-ssp \\
\2         security-classification \3 Enumerated, -- -- unmarked/restricted \\
\2         privacy-mark \3            Printable String -- -- ``a'',``b'',``c'' \\
\2         \}
\\
\etab
}

The simple security policy of {\em DFN} for electronic mail,
identified by the
object identifier {\em id-dfn-ssp} (see paragraph \ref{asn1-oid}),
defines two security classifications for messages,
one called {\em unmarked}, the other one {\em restricted}.
For unmarked messages, no further security element is included
at message composition, and no security element is evaluated
at message reception.
Restricted messages allow for three different privacy marks:
{\em authentic}, {\em confidential}, or {\em both, authentic and confidential}.
These three privacy marks are represented by the one-character
printable strings ``a'', ``c'', or ``b'', respectively.

It shall be allowed that a security classification says {\em confidential}.
This shall be regarded as equivalent with the classification
{\em restricted} and the privacy mark set to ``c''.
If, additionally, the privacy mark is set to ``a'' or ``b'',
the message shall be provided with the integrity service as well.

There is no further security category.

\subsubsection{Message Token}
\label{asn1-token}

The X.400 message token consists
mainly of the two portions ``signed data'' and ``encrypted data''.
The ``recipient name'' and the encrypted ``content confidentiality key''
allow for message content encryption on a per-recipient base.

{\small
\btab
\1 message-token ::= {\bf SEQUENCE} \{ \\
\2 type  \3 [0] INTEGER, -- -- 16 \\
\2 value \3 [2] MessageToken \} \\
\\
\1 MessageToken ::= {\bf SEQUENCE} \{ \\
\2 token-type-identifier \3 [0] OBJECT IDENTIFIER, \\
\2                       \3 -- -- id-tok-asymmetricToken \\
\2 token \3 [1] AsymmetricToken \} \\
\\
\1 AsymmetricToken ::= {\bf SIGNED SEQUENCE} \{  \\
\2 signature-algorithm-id. \3 AlgorithmIdentifier, -- -- Hash with RSA \\
\2 recipient-name \3 ORAddressAndOrDirectoryName, \\
\2 time  \3 UTCTime, \\
\2 signed-data \3 [0] TokenData, \\
\2 encryption-algorithm-id. \3 [1] AlgorithmIdentifier, -- -- RSA \\
\2 encrypted-data \3 [2] ENCRYPTED TokenData \} \\
\\
\etab
}
{\small
\btab
\1 TokenData ::= {\bf SEQUENCE} \{ \\
\2 type  \3 [0] INTEGER, -- -- 1 for ``bind token'' \\
\2 value \3 [1] BitString \} -- -- Random Number \\
\\
\1 TokenData ::= {\bf SEQUENCE} \{ \\
\2 type  \3 [0] INTEGER, -- -- 2 for ``signed'' \\
\2 value \3 [1] MessageTokenSignedData  \} \\
\\
\1 TokenData ::= {\bf SEQUENCE} \{ \\
\2 type  \3 [0] INTEGER, -- -- 3 for ``encrypted'' \\
\2 value \3 [1] MessageTokenEncryptedData  \} \\
\\
\etab
}
{\small
\btab
\1 MessageTokenSignedData ::= {\bf SEQUENCE} \{ \\
\2 content-confidentiality-algo- \\
\2 rithm-identifier        \3 [0] Algorithm Identifier, -- -- DES \\
\2 message-security-label  \3 [2] SecurityLabel, \\
\2 message-sequence-number \3 [4] INTEGER OPTIONAL \\
\2 \} -- -- signed by the originator of the message \\
\\
\1 MessageTokenEncryptedData ::= {\bf SEQUENCE} \{ \\
\2 content-confidentiality-key  \3 [0] BIT STRING -- -- DES-key \\
\2 \} -- -- encrypted for the recipient of the message \\
\\
\etab
}

ORAddressAndOrDirectoryName is defined in X.411, Fig. 2.

\subsubsection{Message Confidentiality Elements}
\label{asn1-conf}

The X.400 extension element
{\em content confidentiality algorithm identifier}
is related to an encryption algorithm
by which the content of a confidential message is encrypted.

{\small
\btab
\1 content-confidentiality-algorithm-identifier ::= {\bf SEQUENCE} \{ \\
\2         type  \3 [0] INTEGER, -- -- 17 \\
\2         value \3 [2] AlgorithmIdentifier \} \\
\\
\etab
}

In practice this should be the algorithm identifier of
the symmetric algorithm.
The symmetric key used to encrypt and decrypt the message content
is encrypted inside the message token encrypted data:

{\small
\btab
\1 content-confidentiality-key ::= [0] {\bf BitString} \\
\\
\etab
}
In case of the DES algorithm, for example, the BitString contains the 64 bits of the DES-key,
whereby the parameter of the DES algorithm identifier
is either omitted or is an OctetString containing 8 octets
of an initialization vector.

\subsubsection{Message Integrity Elements}
\label{asn1-integrity}

The X.400 extension element
{\em message origin authentication check}
contains a signature of the algorithm identifier
by which the signature of the message content is performed,
of the message content,
and of the security label.

{\small
\btab
\1 message-origin-authentication-check ::= {\bf SEQUENCE} \{ \\
\2         type  \3 [0] INTEGER, -- -- 19 \\
\2         criticality  \3 [1] BitString, -- -- 2 for delivery \\
\2         value \3 [2] MessageOriginAuthenticationCheck \} \\
\\
\1 MessageOriginAuthenticationCheck ::= {\bf SIGNATURE SEQUENCE} \{ \\
\2 algorithm-identifier \3   AlgorithmIdentifier, -- -- Hash with RSA \\
\2 content \3                Content, \\
\2 message-security-label \3 SecurityLabel \} \\
\2 -- -- signed by the originator of the message
\etab
}

The X.400 extension element
{\em message integrity check}
is not used in this context.

\subsubsection{Bind Parameter Strong Credentials}
\label{asn1-cred}

{\small
\btab
\1 StrongCredentials ::= {\bf SET} \{ \\
\2 bind-token \3 [0] Token \} \\
\\
\1 Token ::= {\bf SEQUENCE} \{ \\
\2 token-type-identifier \3 [0] OBJECT IDENTIFIER, \\
\2                       \3 -- -- id-tok-asymmetricToken \\
\2 token \3 [1] AsymmetricToken \} \\
\\
\1 AsymmetricToken ::= {\bf SIGNED SEQUENCE} \{  \\
\2 signature-algorithm-id. \3 AlgorithmIdentifier, -- -- Hash with RSA \\
\2 recipient-name \3 ORAddressAndOrDirectoryName, \\
\2 time  \3 UTCTime, \\
\2 signed-data \3 [0] TokenData \} \\
\\
\1 TokenData ::= {\bf SEQUENCE} \{ \\
\2 type  \3 [0] INTEGER, -- -- 1  \\
\2 value \3 [1] BitString \} -- -- Random Number \\
\\
\etab
}

ORAddressAndOrDirectoryName is defined in X.411, Fig. 2.
\\ [1em]
Before the first association,
two partner agents exchange their user certificates
and verify them completely up to a commonly
trusted certification authority.
Therefore, the user certificates need not to be exchanged
during the bind operation.
For partner authentication,
the signature of a token is verified by a trusted public key
already known to the verifier.
Also, the recipient name, time and random number are checked
for validity.
The random number of the responder shall be the same
as the random number of the initiator of an association.

\subsubsection{Bind Parameter Security Context}
\label{asn1-context}

{\small
\btab
\1 SecurityContext ::= {\bf SET} \{ SecurityLabel \} \\
\\
\1 SecurityLabel ::= {\bf SET} \{ \\
\2   security-policy-identifier \3 OBJECT IDENTIFIER -- -- id-dfn-ssp \\
\2   \}
\etab
}

The security context demands for strong partner authentication
with the strong credentials described in paragraph \ref{asn1-cred} above.
\\
In an extended version,
the security classification {\em confidential} indicates
that all parameters of the subsequent operations of this association
are encrypted.
The encryption key will be transferred
within the bind token encrypted data
in the strong credentials
from the initiator to the responder of the association.

\subsection{ASN.1 Structures of the PSE Objects}
\label{asn1-sw-cc}

\subsubsection{KeyBits}
\label{asn1-KeyBits}

Key information data will be encoded as BitStringS,
like in Subject\-Public\-Key\-Info, or in SubjectPrivateKeyInfo
(see \ref{asn1-SignatureSK} and \ref{asn1-DecryptionSKnew} below).
However, the bitstrings themselves will have a substructure
as ASN.1-type SEQUENCE of two integers:

{\small
\begin {center}
\begin {tabular}{lll}
KeyBits ::= & \multicolumn{2}{l} {{\bf SEQUENCE} \{} \\
	    & part1 & INTEGER    \\
	    & part2 & INTEGER \}
\end {tabular}
\end {center}
}

In case of a secret RSA-key,
the BitString of subjectPrivateKey
contains the ASN.1 encoding of a sequence of
the factors p and q of the RSA-modulus n.
In case of a public RSA-key,
which is going to be transferred inside of certificates,
the BitString of subjectPublicKey
contains the ASN.1 encoding of a sequence of
the RSA-modulus n and the public RSA-exponent e,
as it is defined by X.509, Annex C, C.3.
The corresponding C-structure is a structure of two
C-structures OctetString.
\\[1ex]
Encoding/Decoding: e\_KeyBits() and d\_KeyBits()
control the substructures of key information BitStrings.
These substructures are {\em only}
used {\em internally} by the interface functions
{\em sec\_encrypt, sec\_decrypt, sec\_sign, sec\_verify}.
An application programmer will not need to handle them.

\subsubsection{Name}
\label{asn1-Name}
                
{\em Name} is defined in X.501, Annex C, as a SEQUENCE of Relative
Distinguished Names, which are sets of Attribute Value Assertions (see~\ref{asn1-name}).
The object {\em Name} denotes the owner of the PSE.

\subsubsection{SignSK}
\label{asn1-SignatureSK}

{\small
\begin {center}
SignSK ::= SubjectPrivateKeyInfo
\end {center}
}

{\em SubjectPrivateKeyInfo} is defined to have the same format as
{\em SubjectPublicKeyInfo}, cf. X.509, Annex G.
However, note the substructure of the private key BIT STRING.

{\small
\begin {center}
\begin {tabular}{lll}
SubjectPrivateKeyInfo ::= & {\bf SEQUENCE} \{ & \\
  & algorithm         & AlgorithmIdentifier,    \\
  & subjectPrivateKey & BitString \}           \\
  & \multicolumn{2}{l} {-- --} \\
  & \multicolumn{2}{l} {-- -- the BitString of subjectPrivateKey}   \\
  & \multicolumn{2}{l} {-- -- contains the ASN.1 encoding of a sequence of} \\
  & \multicolumn{2}{l} {-- -- the factors p and q of the RSA-modulus n:}   \\
  & \multicolumn{2}{l} {-- --} \\
  & -- -- SEQUENCE \{ & p INTEGER,   \\
  & -- --             & q INTEGER \}
\end {tabular}
\end {center}
}
 
Encoding/Decoding: e\_KeyInfo(), d\_KeyInfo().

\subsubsection{DecSKnew}
\label{asn1-DecryptionSKnew}

{\small
\begin{center}
DecSKnew ::= SubjectPrivateKeyInfo
\end{center}
}

{\em SubjectPrivateKeyInfo} is defined in {\em SignSK}
(see \ref{asn1-SignatureSK} above).
\\[1ex]
Encoding/Decoding: e\_KeyInfo(), d\_KeyInfo().

\subsubsection{DecSKold}
\label{asn1-DecryptionSKold}

As {\em DecSKnew} (see \ref{asn1-DecryptionSKnew} above).

\subsubsection{SignCert}
\label{asn1-SignCertificate}

{\small
\begin{center}
SignCert ::= Certificate
\end{center}
}

For the definition of {\em Certificate} see under 
directory attribute {\em Certificate} in~\ref{asn1-cert}.
\\ [1em]
Encoding/Decoding: e\_Certificate(), d\_Certificate().

\subsubsection{EncCert}
\label{asn1-EncrCertificate}

As {\em SignCert} (see \ref{asn1-SignCertificate} above).

\subsubsection{Certificate Sets}
\label{asn1-certificateset}
{\small
\bc
        SignCSet ::= {\bf SET OF} Certificate \\
        EncCSet ::= {\bf SET OF} Certificate
\ec
}
For each type of certificate exists one certificate set collecting all
certificates which have been issued for the same public key. \\ [1ex]
Encoding/Decoding: e\_CertificateSet(), d\_CertificateSet()

\subsubsection{FCPath}
\label{asn1-FCPath}

{\small
\begin {center}
\begin {tabular}{lll}
FCPath ::= & \multicolumn{2}{l} {
		 {\bf SEQUENCE OF} CrossCertificates} \\[1ex]
CrossCertificates ::= & \multicolumn{2}{l} {{\bf SET OF} Certificate}
\end {tabular}
\end {center}
}

{\em FCPath} is an element of {\em Certificates} (note the plural).
{\em Certificates} is defined in X.509, Annex G,
as a SEQUENCE of a user certificate and a forward certification path.
The forward certification path is a SEQUENCE of cross certificates,
which are SETs of 1-level certificates.
One cross certificate per SET is supposed to be
the hierarchy certificate of the path.
The data of the PSE object {\em FCPath}
can be composed both with the data of
{\em SignCert} or of
{\em EncCert} in order to build an
n-level originator certificate of ASN.1-type {\em Certificates}.

{\small
\begin {center}
\begin {tabular}{lll}
Certificates ::= & {\bf SEQUENCE} \{  & \\
  & certificate          & Certificate,                \\
  & certificationPath    & FCPath    \\
  &                      & OPTIONAL \} \\[1ex]
\end {tabular}
\end {center}
}

Encoding/Decoding: e\_FCPath(), d\_FCPath().

\subsubsection{PKRoot}
\label{asn1-PKRoot}

{\small
\begin {center}
\begin {tabular}{lll}
PKRoot ::= & {\bf SEQUENCE} \{ &    \\
  & cA                & Name,                 \\
  & serialNew         & INTEGER,              \\
  & keyNew            & SubjectPublicKeyInfo, \\
  & serialOld         & INTEGER,              \\
  & keyOld            & SubjectPublicKeyInfo \}
\end {tabular}
\end {center}
}
 
{\em Name} is defined in X.501, Annex C,
as a SEQUENCE of Relative Distinguished Names,
which are sets of Attribute-Value-Assertions.
{\em SubjectPublicKeyInfo} is defined by X.509, Annex G
(see {\em Certificate} in \ref{asn1-cert} above).
\\[1ex]
Encoding/Decoding: e\_PKRoot(), d\_PKRoot().

\subsubsection{PKList}
\label{asn1-PKList}

{\small
\begin {center}
\begin {tabular}{lll}
PKList ::= {\bf SET OF} & {\bf SEQUENCE} \{ & \\
 & version [0]     & INTEGER DEFAULT 0,       \\
 & serialNumber    & INTEGER,                 \\
 & signature       & AlgorithmIdentifier,     \\
 & issuer          & Name,                    \\
 & validity        & Validity,                \\
 & partner         & Name,                    \\
 & verificationKey & SubjectPublicKeyInfo \}
\end {tabular}
\end {center}
}
 
Compare the definition of {\em Certificate} in~\ref{asn1-cert} above. A PKList is defined
as a set of {\em ToBeSigned} values of certificates. \\ [1ex] 
The {\em verificationKey} is supposed to have the value
of the currently valid public verification key of the {\em partner}.
Before adding it to this list, it should be verified through
the full certification path.
\\[1ex]
Encoding/Decoding: e\_PKList(), d\_PKList().

\subsubsection{EKList}
\label{asn1-EKList}

{\small
\begin {center}
\begin {tabular}{lll}
EKList ::= & {\bf SET OF SEQUENCE} \{ &  \\
 & version [0]     & INTEGER DEFAULT 0,       \\
 & serialNumber    & INTEGER,                 \\
 & signature       & AlgorithmIdentifier,     \\
 & issuer          & Name,                    \\
 & validity        & Validity,                \\
 & partner         & Name,                    \\
 & encryptionKey   & SubjectPublicKeyInfo \}
\end {tabular}
\end {center}
}
 
{\em EKList} has the same format as {\em PKList}
(see \ref{asn1-PKList} above).
However,
the {\em encryptionKey} is supposed to have the value
of the currently valid public encryption key of the {\em partner}.
Before adding it to this list, it should be verified through
the full certification path.
\\[1ex]               

Encoding/Decoding: e\_PKList(), d\_PKList().

\subsubsection{Toc}
\label{asn1-TOC}

{\small
\begin {center}
\begin {tabular}{lll}
Toc ::= & {\bf SEQUENCE} \{ &  \\
  & owner         & PrintableString, \\
  & create        & UTCTime,  \\
  & update        & UTCTime,  \\
  & PSE\_Objects   & objects \} \\ \\ \\
PSE\_Objects ::= & {\bf SET OF} PSE\_Object & \\ \\ \\
PSE\_Object ::= & {\bf SEQUENCE} \{ &  \\
  & name          & PrintableString, \\
  & create        & UTCTime,  \\
  & update        & UTCTime \}
\end {tabular}
\end {center}
}
 
Encoding/Decoding: e\_PSEToc(), d\_PSEToc().
