\documentstyle[11pt,dina4,twoside]{article}
\marginparsep5mm
\parskip 1ex

%
% Definition eigener Makros fuer SECUDE
%
% einbetten mit \input{spec-makro}
%
\def\parname{\parbox[t]{3.3cm}}
\def\pardescript{\parbox[t]{11.6cm}}
\def\firstbox{\parbox[t]{5.3cm}}
\def\secondbox{\parbox[t]{9.6cm}}
\def\parba{\parbox[t]{4cm}}
\def\parbe{\parbox[t]{11cm}}
\newcommand{\speconly}[1]{#1}
\newcommand{\manonly}[1]{}
\def\nopagenumbering{
        \pagestyle{empty}
        \pagenumbering{alph}
        \setcounter{page}{0}
        \thispagestyle{empty}
}
\def\bi{\begin{itemize}}
\def\ei{\end{itemize}}
\def\bc{\begin{center}}
\def\ec{\end{center}}
\def\be{\begin{enumerate}}
\def\ee{\end{enumerate}}
\def\bd{\begin{deflist}}
\def\ed{\end{deflist}}
\def\bv{\begin{verbatim}}
\def\ev{\end{verbatim}}
\def\bquote{\begin{quote}}
\def\equote{\end{quote}}
\def\m{\item}
\def\n{\newpage}
\def\ka{\raisebox{.4ex}{{\tiny $<\!\!\!<$}}}
\def\kz{\raisebox{.4ex}{{\tiny $>\!\!\!>$}}}
\def\marker#1{\marginpar{\rule[0cm]{1mm}{#1}}}
\def\pf{\-{--\raisebox{.25ex}{{\scriptsize $>$}}}}
\def\btab{
        \begin{tabbing}
        aaaaaaaa \= aaaaaaaa \= aaaaaaaa \= aaaaaaaa \= aaaaaaaa \= aaaaaaaa \= aaaaaaaa \= aaaaaaaa \= aaaaaaaa \= aaaaaaaa \= \kill
}
\def\bvtab{
        \begin{tabbing}
        aaaa \= aaaa \= aaaa \= aaaa \= aaaa \= aaaa \= aaaa \= aaaa \= aaaa \= aaaa \=aaaa \=aaaa \= \kill
}
\def\etab{\end{tabbing}}
\def\evtab{\end{tabbing}}
\def\1{\>}
\def\2{\> \>}
\def\3{\> \> \>}
\def\4{\> \> \> \>}
\def\5{\> \> \> \> \>}
\def\6{\> \> \> \> \> \>}
\def\7{\> \> \> \> \> \> \>}
\def\8{\> \> \> \> \> \> \> \>}
\def\9{\> \> \> \> \> \> \> \> \>}
\def\blankpage{
        \nopagenumbering
        \newpage
        \btab
        \etab
        \newpage
}
\newcommand{\hll}[1]{}
\newcommand{\hl}[1]{\vs{0.5cm} {\bf #1} \\ [1ex]}
\newcommand{\nm}[3]{#3}
\def\addtotoc#1{}
\def\vs#1{
        \mbox{} \vspace{#1} \\
}
\def\ifnot#1#2{
        \def\firstpar{#1}
        \def\secondpar{#2}
        \ifx\firstpar \secondpar \else \secondpar \\ \fi
}
\def\eline#1{
        \def\fpar{#1}
        \def\spar{}
        \ifx\fpar \spar \else \fpar \\ [1cm]\fi
}
\def\secudetitlepage#1#2{
        \title{ {\Huge\bf  SecuDE} \\ [1.5cm]
                {\Large\bf #1} \\ [0.5cm] 
                {\large \eline{#2}}
                {\large Version 4.1} \\ [2cm]
        }
        \author{
                \authors{}
                GMD \\
                Institut f{\"u}r TeleKooperationsTechnik (I2) \\
                Darmstadt, Germany \\ [5cm] 
        }
        \date{
                \today \\ [1cm]
                {\normalsize Copyright \copyright 1993 by Gesellschaft f{\"u}r Mathematik und Datenverarbeitung (GMD)}
        }
        \nopagenumbering
        \maketitle
        \parindent0em
        \newpage
        \conditions{}
        {\small
        Other volumes available within this documentation: \\ [1ex]
	\ifnot{#1}{Overview}
	\ifnot{#1}{Vol. 1: Principles of Security Operations}
	\ifnot{#1}{Vol. 2: Security Commands, Functions and Interfaces}
 	\ifnot{#1}{Vol. 3: Security Applications' Guide}
        }
        \newpage
	\pagenumbering{roman}
        \preface{}
	\cleardoublepage
}

\def\conditions{
        Copyright \copyright Gesellschaft f{\"u}r Mathematik und Datenverarbeitung (GMD), 1993
        \\ [1cm]
        {\footnotesize
        Permission to use, copy, modify, and distribute this software and its documentation
        for any purpose and without fee is hereby granted, provided that this notice and the
        reference to this notice appearing in each software module be retained unaltered, and 
        that the name of GMD or any contributor shall not be used in advertising or publicity 
        pertaining to distribution of the software without specific written prior permission. 
        It is the responsibility of the users of this software to comply with national or 
        international export and import regulations, or with licence rights from third parties.
        \\ [1em]
        GMD AND ALL CONTRIBUTORS DISCLAIM ALL WARRANTIES WITH REGARD TO THIS SOFTWARE, INCLUDING
        ALL IMPLIED WARRANTIES OF MERCHANTIBILITY AND FITNESS. IN NO EVENT SHALL GMD OR ANY
        CONTRIBUTOR BE LIABLE FOR ANY SPECIAL, INDIRECT OR CONSEQUENTIAL DAMAGES OR ANY DAMAGES
        WHATSOEVER RESULTING FROM LOSS OF USE, DATA OR PROFITS, WHETHER IN AN ACTION OF CONTRACT,
        NEGLIGENCE OR OTHER TORTIOUS ACTION, ARISING OUT OF OR IN CONNECTION WITH THE USE OR
        PERFORMANCE OF THIS SOFTWARE. \\ [8cm]
        }
}

\def\preface{
	{\Large\bf Preface}
	\\ [1cm]
        SecuDE has been developed by GMD for the R \& D community as a research tool to
        promote the construction of trustworthy systems, particularly in the context of OSI.
        The initial sponsorship came from the Deutsches Forschungsnetz (DFN), project 
        {\em Secure DFN}, which was made possible by grants from the Ministry of Research 
        and Technology of Germany. 
        \\ [1em]
	The SecuDE documentation was prepared by R{\"u}diger Grimm, Jan L{\"u}he and 
        Wolfgang Schneider.
        \\ [1em]
	The following people contributed to the software development of SecuDE: 

	R{\"u}diger Grimm, Stephan Kolletzki, Jan L{\"u}he, Rolf-Dieter Nausester, 
        J{\"o}rg Reichelt, Wolfgang Schneider, Thomas Surkau and Ursula Viebeg contributed 
        to SecuDE in general. 
	The arithmetic package and the RSA and DSA software were written by Achim Jung, 
        Wolfgang Bott, Rolf-Dieter Nausester, Thomas Surkau and Stephan Thiele. 
        The STARCOS smartcard package and its integration into SecuDE was done by Levona 
        Eckstein, Helga Parslow, Ursula Viebeg and Zhou Anmin.
	\\ [1em]
        The SecuDE package includes parts of ISODE Release 8.0 ASN.1 and QUIPU tools, 
        Phil Karn's DES Package, MD2, MD4 and MD5 software from RSA Data Security Inc., 
        and a SHA implementation from Peter C. Gutman and Colin Plumb (pgut1@cs.aukuni.ac.nz). 
        \\ [1em]
        Comments are welcome and should be addressed to: 

        \parba {Postal Address:}
        \parbe {
                GMD, Institut f{\"u}r TeleKooperationsTechnik (I2) \\
                Dolivostr. 15 \\
                6100 Darmstadt \\
                Germany \\
        }
        \parba {Telephone:}
        \parbe {
                +49-6151-869-719 or -718
        }
        \parba {Fax:}
        \parbe {
                +49-6151-869-785 \\
        }
        \parba {E-Mail:}
        \parbe {
                Internet-Address: \\
                schneider@darmstadt.gmd.de \\
                luehe@darmstadt.gmd.de \\ [1ex]
                X.400-Address: \\
                C=DE; ADMD=DBP; PRMD=GMD; OU=Darmstadt; S=Schneider
        }
	\newpage
	{\Large\bf How to Get SecuDE}
	\\ [1em]	
        The most recent version of SecuDE can be obtained by 
	\bi
        \m ANONYMOUS FTP from internet address {\em ftp.darmstadt.gmd.de} (141.12.50.1), 
        \m FTAM (User-Id {\em anon}) from X.25 address 262 45050363363.
	\ei 
        cd to the pub/secude directory to find 
	\bi
        \m three .ps.Z files comprising compressed postscript files of the available
           documentation,
        \m the file secude-x.y.all.tar.Z comprising a tar'ed and compressed directory 
           of everything including program sources and documentation sources (latex),
        \m the file secude-x.y.src.tar.Z comprising the program sources only,
        \m the file secude-x.y.doc.tar.Z comprising the documentation sources only.
	\ei
	x.y is currently 4.1.
	\\ [1cm]
	{\Large\bf Electronic Mail Distribution List}
	\\ [1em]
	For electronic mail discussions about SecuDE use the Internet address
	
	{\em secude@darmstadt.gmd.de}
	
	To get added to the list send an informal mail to {\em secude-request@darmstadt.gmd.de}. 
	\\ [1cm]	
	{\Large\bf Restrictions}
	\\ [1em]
	The SecuDE package contains cryptographic software for both authentication and
        encryption purposes. This type of software might be subject to national or
        international export or import regulations. It contains also software which
	implements the RSA algorithm. The RSA algorithm is patented in USA.
	
	Those who obtain the SecuDE software via public network access are advised to get
        acquainted with the respective regulations and licence conditions for that
	environment where they intend to use the software. It is their responsibility
	not to get in conflict with such regulations.
}

\def\authors{
Edited by Wolfgang Schneider \\ \\
}
\begin{document}

\secudetitlepage{Overview}{}

\pagestyle{plain}
\pagenumbering{roman}
\tableofcontents

\cleardoublepage

\newcounter{Abb}
\pagenumbering{arabic}
\markboth{SecuDE Overview}{SecuDE Overview}

\pagestyle{myheadings}
\thispagestyle{myheadings}

\section{Summary}
\subsection{General}
The contribution of SecuDE to public domain is part of our effort to 
facilitate the open, authentic and privacy-preserving electronic 
telecooperation between people.  
Authenticity and protection of privacy is an increasing concern of 
everyone as electronic information storage and exchange is rapidly 
growing. It can be credibly achieved only with publicly known
security technology of which the effectiveness is understandable and 
comprehensible through public control. 
Security must only rely on the secrecy of keys, not on the secrecy of methods.
The use of public-key cryptography is another cornerstone which
makes authenticity achievable and manageable in a large scale open 
electronic communication society. 

SecuDE (Security Development Environment) is a security toolkit for
Unix systems which incorporates well known and established symmetric
and public-key cryptography. 
It offers a library of security functions and a
number of utilities with the following functionality:
\bi
\m basic cryptographic functions like RSA, DSA, DES, various hash functions, 
   including the RFC 1423 defined algorithm suite, 
   OIW defined algorithms, and DSS,

\m security functions for {\em origin authentication}, {\em data integrity}, 
   {\em non-repudation of origin} and {\em data confidentiality} purposes 
   on the basis of digital signatures and symmetric and asymmetric encryption,

\m X.509 key certification functions, handling of certification pathes, 
   cross-certification, certificate revocation,

\m utilities and functions for the operation of certification authorities 
   (CA) and interaction between certifying CAs and certified users,

\m utilities to sign, verify, encrypt, decrypt and hash files,

\m Internet PEM processing according to RFC 1421 - 1424 (see~\ref{pem})

\m processing of RFC 1422-defined certificate revocation lists,

\m secure access to public X.500 security attributes for the storage 
   and retrieval of certificates, cross-certificates and revocation 
   lists (integrated secured DUA, based on QUIPU 8.0, using strong 
   authentication and signed DAP operations, see~\ref{x500}),

\m all necessary ASN.1 encoding/decoding (based on Isode-8.0)

\m integrity-protected and confidentiality-protected storage of all security 
   relevant information of a user (secret keys, verification keys, certifiactes
   etc.) in a so called {\em Personal Security Environment} (PSE)
\footnote{
For example, a PSE typically contains the user's secret and public key (the latter
contained in an X.509 certificate), the public root key which the user
trusts, the user's distinguished name, the user's login name, and the 
forward certification path to the user's root key. In addition, the PSE 
allows to securely store other's public keys after their validation 
(allowing henceforth to trust them like the root key without verifying 
them again), and certificate revocation lists (CRLs).}
.
\ei
 
SecuDE provides two different PSE realizations,
\bi 
\m a smartcard environment (SC-PSE), 
\m a DES-encrypted Unix directory (laxly called software-PSE or SW-PSE),
\ei 
both only accessible through the usage of PINs (Personal Identification Numbers).
Smartcards require the purchase of a particular smartcard environment where 
RSA and DES cryptography is done in the smartcard reader (information available 
on request).

\subsection{SecuDE Privacy Enhanced Mail}
\label{pem}
An Internet Privacy Enhanced Mail implementation (PEM RFC 1421-1424) 
is part of SecuDE. It provides a PEM filter
which transforms any input text file into a PEM formatted output file 
and vice versa, and which should be capable of being  easily 
integrated into Mail-UAs \footnote{An MH-6.8 version with integrated 
SecuDE-4.1 PEM filter will be available from GMD
in a separate distribution.}
or CA tools. SecuDE-PEM realizes all formats 
and procedures defined in the Internet Specifications RFC 1421-1424 
except that it only supports asymmetric key management. It is
possible to securely cache other's certificates and CRLs as this 
is part of the general SecuDE functionality.

SecuDE-PEM supports the certification and CRL procedures defined in 
RFC 1424 and is integrated into the SecuDE CA functionality. As an additional 
functionality which goes beyond RFC 1421 - 1424, SecuDE-PEM may be 
configured with an integrated X.500 DUA which allows, for instance, 
automatic retrieval of certificates and CRLs during the PEM de-enhancement process.


\subsection{SecuDE X.500 Directory Integration}
\label{x500}
Since the use of public key cryptography and certification makes it necessary
to deal with public security information, the integration of public Directories 
(in particular X.500 Directories) into the operation of security procedures
is an essential part of SecuDE.

Concerning X.500 Directory access, SecuDE can be configured in three 
alternatives:
\be
\m Without X.500 DUA functionality. In this case, the SecuDE package
     is self-contained and needs no additional software.

\m With integrated X.500 DUA, which operates on the basis of simple
     authentication (i.e. with passwords) and unprotected DAP operations.
     SecuDE uses QUIPU-8.0 library functions for this DUA functionality.
     These library functions are not included in the SecuDE package, and
     a standard QUIPU-8.0 installation is additionally required (the Isode/-
     QUIPU libraries libisode.a and libdsap.a are needed to be able to bind 
     the application programs).

\m With integrated X.500 DUA, which operates on the basis of strong authen-
     tication (i.e. with digital signatures) and signed DAP operations.
     SecuDE uses modified (i.e. security-enhanced) QUIPU-8.0 library 
     functions for this DUA functionality. These library functions are not 
     included in the SecuDE package, and a security-enhanced QUIPU-8.0 
     installation is additionally required (the security-enhanced Isode/QUIPU 
     libraries libisode.a and libdsap.a are needed to be able to bind the 
     application programs).
\ee
Alternative 3 requires a security-enhanced version of Quipu which can be
obtained as a separate distribution from GMD. 

The security-enhanced version of Quipu also contains software for a secured X.500 
DSA. "Secured" means the ability to perform strong authentication during
association setup between DUA and DSA, and subsequent signed operations at the 
DAP level. The DAP operations into which we have incorporated strong authentication 
are those assigned by the standard (X.511) for that purpose, i.e. Bind, Read, 
Compare, Search, List, AddEntry, RemoveEntry, ModifyEntry, and ModifyRDN. We have 
provided both SIGNED arguments and SIGNED results.

\subsection{Use of Smartcards}
   SecuDE-4.1 supports the use of the GAO/GMD smartcard package Starcos 1.0 as
   one realization of the PSE. A serial line interface with 19.200 Baud is required
   to connect the Starcos terminal to the workstation. The Starcos terminal
   is a high-security crypto device which performs RSA and DES and provides a number of 
   physical protection features. RSA key pairs are generated in the Starcos terminal. 
   Secret RSA keys never leave the terminal.

\section{A Brief Tutorial to Security}
\label{tutorial}
\markboth{Security Tutorial}{Security Tutorial}
\pagestyle{myheadings}
Since computers and computer communication are increasingly being used to store and
exchange assets, measures have to be introduced to protect those assets
against accidental or intentional loss, alteration and misuse. The assets to be protected
can be of very different nature. They may consist of system resources owned by service providers
who offer access to data bases, information systems or general computer services, they may comprise
information which is stored and exchanged with the help of computers, and they may even comprise 
electronically signed and stored contracts by which other parties can be held responsible, or
other electronically processed information which guarantees bindings of responsibilities
or legal bindings in organizational contexts. 

The latter examples show that new security techniques are not only needed in areas
where classical security means as the usage of passwords have been proven to be 
unsufficient. Security is needed for areas where the application of computer
communication is going to: as a means for cooperative work which may
span over organizations between an open number of partners who may have 
conflicting interests and who possibly don't know each other when the communication
starts. For this area, known as {\em open telecooperation}, 
advanced security techniques are a necessity. 

Security is needed by both users of computer systems and providers of computer
services. The classical way of achieving security reflects more to the interests
of system providers than to the interests of users.
First, users are often rigidly restricted to services and system resources 
which they have paid for or which 
system administrators and service providers believe to be sufficient for the user's needs,
second, access to system resources and information is provided by authentication
techniques based on passwords which protect, in the best case, the system
providers and the users among each other, but which do not protect users from
privileged personnel, 
and third, the user's activities are often traced as comprehensively as it is necessary
for the system administrator's needs. System providers and system users, however,
have different interests. Central user identification, service restrictions and
activity traces help the providers to protect their systems and to get their payment,
but they limit the comfort of users and, more seriously, they jeopardize the privacy
of users. This leads to closed systems. On the other hand, open systems which do
not provide the possibility to users to set up closed subenvironments are unusable
for a lot of applications.
Open systems in which users are able to demand services, 
in the ideal case, spontaneously, unlimited, if desired anonymously, without
prior registration and without succeeding system trace, cannot be realized by 
the conventional password oriented means.

\subsection{Key Issues of Security}
We can observe basicly four types of threats against communication security
in open networks which can be qualified as follows:
\bi
\m Unauthorized release of information by passive observation.
\m Unauthorized modification of information, i.e. changing, duplicating
   or replaying exchanged information between two entities.
\m Masquerading in various forms.
\m Repudiation of communication, both by the originator or the
   recipient of communication information.
\ei

{\em Authenticity} is the key issue of security. Authenticity of the subjects (persons,
organizational instances, program instances, hardware components) 
and of the objects (information, files, programs,
keys) of information processing systems is the basis where the transferability
of responsibilities relies upon, which in turn is the basis of every kind
of cooperative work. Security requirements as integrity of data, access control,
non repudiation or prevention of masquerade (i.e. playing the role of others)
can only be served if one is able to rely on authentic partner relations.

{\em Confidentiality} is an issue, too, for a number of applications.
This is the classical playground of cryptography. Information which belongs
to the privacy of people like personal data, medical data etc., 
or information which represent an economic asset for individuals or organizations,
should not be transmitted in cleartext over public
networks if the disclosure of such information by unauthorized people
could lead to economic or other damage. 
A new dimension of eavesdropping threats comes through the use
of local area networks. Systems which are connected to the entry points
of public data networks are rarely single computers which can be additionally
protected by physical conditions. In most cases
local area networks or distributed systems are behind the wide area network.
An Ethernet for instance is extremely easy to monitor if one has access to
the coax cable, and the extensive use of distributed file systems like NFS
leads to the situation that often one even doesn't know what heavy
network activities are triggered by a command which looks like a local
file access at the operating system interface. This example also shows
that standard measures against eavesdropping as line encryption
wouldn't help here. End-to-end encryption between the connected applications 
is essential in many cases, while encryption measures betwen network
components are additional possibilities which might be useful to
encounter traffic analysis and the like.

However, it should be noted that achieving confidentiality is not possible without
authenticity. One must be sure with whom to exchange or share keys for confidentiality
purposes, otherwise encryption might not be of great value. Authenticity is a
prerequisite for other security services and therefore the key issue of security.

To summarize, we can observe a need for
\bi
\m the possibility of authenticly identifying acting persons or instances in 
   a decentralized and user controlled way which requires the disclosure of
   not more individual information than necessary in order to satisfy the legitimate
   interests of all concerned parties, and which avoids any unnecessary involvement
   and traces of central authorities,
\m the possibility of providing and verifying a proof of authenticity and integrity
   of information,
\m and the possibility of guaranteeing confidentiality and privacy in
   a multi-party environment.
\ei

\subsection{Cryptography}
The basic means to provide security of the nature mentioned above is cryptography.
Cryptographic algorithms are being used for two purposes. One is to prove to others that
one is in the possession of a certain key. If the partner or verifier is able to decrypt
an information which was previously encrypted using this key, the proof is provided.
If it is assured by other means, in addition, that no other person or instance can
be in possession of this key, the use of this key produces a link to the user of the key.
This is the way authentication is done. The other is to use cryptography in order
to conceal information, i.e. to avoid unauthorized disclosure of information.
This is the classical application of cryptography which aimes at achieving confidentiality. 

We can distinguish two main categories of cryptographic algorithms. One category
comprises the {\bf symmetric} algorithms where encryption and corresponding decryption is performed 
using the same key, but inverse functions. The best known symmetric algorithm is the
{\em Data Encryption Standard} (DES) invented 1977 by the American NBS (National Bureau of
Standards, now NIST) and internationally standardized as DEA1 (Data Encryption Algorithm 1).

The second category comprises the {\bf asymmetric} algorithms. This type of algorithms was invented by Diffie and Hellmann in 1976 
and uses different keys for encryption
and corresponding decryption. The two keys are mathematically dependent
from each other, but it is a requirement to the algorithm that one key can be computed 
from the other key at most in one direction, while the other direction must be 
computationally unfeasable. This property allows to make one key 
(the one which can be derived from the other) public while the other key must be kept secret
and unbreakably linked to a person or instance. Therefore this type of algorithm is also
called {\bf public key} algorithm. 

The class of asymmetric algorithms can be subdivided into the two categories {\bf reversible}
public key algorithms and {\bf irreversible} public key algorithms.  Reversible algorithms
have also the property that applying the encryption function to a cleartext followed by 
applying the decryption function to the ciphertext results in the cleartext again and has
the same effect as doing it in the reverse order, i. e. we have
\begin{center}
$d(e(cleartext, E), D) = e(d(cleartext, D), E) = cleartext$
\end{center}
{\small
\bvtab
\1 with \3 e: encryption function \7 E: encryption key \\
\4         d: decryption function \7 D: decryption key
\evtab
}
This property allows to use a single algorithm and even a single key pair for both
authentication/integrity and confidentiality puposes in that you use your own secret
key to produce a digital signature which can be verified by everyone through your
public key, and that another person uses your public key to send you information
in a confidential way which you can decrypt using your secret key. 
The best knowm reversible asymmetric algorithm
is the {\em RSA} algorithm invented by R. Rivest, A. Shamir and L. Adleman in 1978.

Irreversible asymmetric algorithms do not have this property, i.e it is not possible
with them to recover the cleartext from the ciphertext, and they don't have encryption / decryption
functions in that sense. With them it is possible to verify with a public key component that
a digital signature was produced with the corresponding secret key component. Therefore this type 
of algorithm is also called {\bf signature-only} algorithm. The best knowm signature-only algorithms
are the {\em ElGamal} algorithm invented by T. Elgamal in 1985, and the NIST variant 
of this algorithm, the {\em DSA} algorithm, used for the proposed US FIPS {\em Digital Signature Standard} (DSS).  

Symmetric algorithms can be realized very efficiently, but they have the problem of requiring
a shared secret. Two partners using symmetric cryptography must have the {\em same} secret
key. This makes symmetric algorithms unsuitable for proving individual identities to 
third parties because at least two share the same key and the knowledge of the key doesn't
provide a link to an individual user. Another problem is the necessity to transmit
symmetric keys accross networks if the communcating partners are on different places.
Symmetric cryptography can be used for the purpose of authentication
in conjunction with a central authentication server whom anyone trusts and who keeps the secrets of
all partners.

The problem of the shared secret does not exist with asymmteric cryptography. Every person
or instance, i.e. every subject,  owns a unique {\em pair} of keys. One of them 
is to be published, while the access to the other must be unbreakably restricted to the
subject itself through a secure local environment, e.g. a smartcard environment.
By means of {\em digital signatures}, which are being generated through encrypting
a hash value of the information to be signed using the secret key, and which can be verified by everyone through decrypting the signature using the public key, authentic identifications 
and proofs of integrity can be exchanged in open systems. 
Confidentiality can be achieved, too, in the case of reversible algorithms (RSA) by encrypting 
the information with the recipient's public
key. This can be done by everyone, but only the recipient is able to decrypt
the information with his secret key. 

The application of asymmetric cryptography leads to a number of practical problems, too.
One of them is that most known asymmetric algorithms rely on
complex mathematical problems of number theory and the usage of very large integer numbers,
which makes them considerably slower than DES, for instance. Thus they are not suitable
for the encryption of large amounts of data. A widely accepted solution is to encrypt
data (for the purpose of confidentiality) using symmetric algorithms and to exchange
the corresponding symmetric keys in encrypted form using reversible asymmetric algorithms.

\subsection{Public Key Certification}
Another fundamental problem, which arises in open environments with large numbers of
partners who do not know each other, is the question of the authenticity of public
keys. At the time of verifying a digital signature the verifier must be sure that the 
public key which he uses for the verification of the signature is the public key of 
the supposed signer, i.e. he needs a key-to-name binding which he can trust. 
Without additional measures each user would have to perform an out-of-band verification
of the authenticity of each partner's public key before trusting it. The complexity
of this problem can be reduced by {\em certifying} public keys through a third party whom
both signer and verifier trust.
This third party, the so called {\em certification authority} (CA),
signs the user's public key and his name (plus some additional
data like a period of validity) with its own secret key. This piece of data, signed  
by the certification authority, is called a {\em certificate}. The certificate can be 
verified with the public key of the certification authority. Now two partners can 
authenticate each other by first verifying the partner's digital signature with the 
partner's public key and then verifying the authenticity of the partner's public
key through verifying the digital signature of the certificate using the public key
of the certification authority. Only the public key of the 
certification authority must be trusted, thus reducing the number of public keys 
which the individual user has to trust to one. 

In large user populations comprising perhaps millions of partners, a single certification 
authority is not sufficient. The public keys of certification authorities may be 
certified again by other certification authorities. One can imagine tree structures 
of certification authorities, or net structures in that CAs belonging to different trees
cross-certify each other, thus providing certification pathes or chains of trust 
between individual partners. 

\subsubsection{Public Root Key}
However, the chain of certificates cannot be endless, and the public key in the last 
certificate remains uncertified. This is called the {\em Public Root Key} which the 
user simply has
to trust finally. The authenticity and integrity of this key has to be provided by 
out-of-band means. The key can be published,  for instance, in a human readable form, 
and particular software may enable the user to compare this key with that what is stored
in his computer. Once the public root key has been entered into the local computer
system, its integrity must be protected by local means (for instance
through the use of smartcards). The public root key, though public, is as security
sensitive as the personal secret key. They are the two ends of the security
chain between two individual partners.

\subsubsection{Certificate Revocation}
A certification authority must be able to revoke a certificate which it issued
prior to its expiration time. There may be many reasons for a certification authority
to revoke a certificate:
\bi
\m The user's secret key is assumed to be compromised whereby the
corresponding public component is invalidated.
\m The user's affiliation has changed whereby the distinguished
name contained in the certificate's "subject" field is invalidated.
\m The user is no longer to be certified by the CA.
\m The CA's certificate is assumed to be compromised.
\m The user has violated the CA's security policy.
\ei

A certification authority can mark a certificate which it issued as "invalid" by adding
it to the list of revoked certificates.

Information relative to certificate revocation is propagated by
means of revocation lists, so-called "black lists".
Revocation lists must be made publicly available, for instance by being placed in
the public X.500 Directory.

\subsection{Key Management}
Key management is the process of achieving suitable keys for the own use
and making them available to the partners in a way that they
are able to validate them and use them in the intended manner, and that
nobody is able to misuse them. This means in the case of symmetric keys 
and asymmetric secret keys that they have to be transmitted 
confidentially (through encryption or out-of-band transport). 
In case of asymmetric public keys additional information (a certificate, for instance) 
must be available to the partner to be able to verify the authenticity of the key.

The same requirements apply for the communication between a certification
authority and its user (which may be a certification authority again)
for the purpose of public key certification. Keys must be exchanged between 
the CA and the user. There are basicly two alternatives for the operation
of public key certification which have different security implications:
\be
\m The CA generates the asymmetric key pair, certifies the public component,
   and provides the user the secret key, the certificate (which includes the
   public key) and possibly own certificates
   which link the whole thing to a common root key.
\m The user generates the asymmetric key pair and sends the public component
   to the CA. The CA certifies this, and provides the user the certificate and possibly 
   own certificates which link the whole thing to a common root key.
\ee
In the first case there is a need for confidentiality-protected information
exchange, in the latter not. Confidential transmission from the CA to the user
is easy if a smartcard is being used which is generated at the CA site and
physically delivered to the user. In case of transmission in electronic form
a symmetric encryption key (or a password where a key can be derived from)
has to be transmitted out-of-band. In the first case, the certification authority
has control over the quality of the generated keys, in the latter case the
user has complete control over his secret key. In both cases, however,
an out-of-band information provision has to take place which gives the CA
the required level of confidence in the user's identity and the name to be certified.
Both types of CA -- user interaction are supported by SecuDE.   

\subsection{Public Directories}
The use of public key cryptography makes it necessary that one knows 
and trusts other's public keys. Public Directories like X.500 Directories
may play a supporting role here. Users and certification authorities
(if they are involved) have the following requirements when they use security
services on the basis of public key cryptography:
\bi
\m Users have the requirement to make their certificates publicly
available and to have access to certificates of others and revocation lists
in order to find out whether a certificate in question is still valid or
has been revoked.
\m Certification authorities have the requirement to make their certificates and revocation
lists publicly available to a distributed community.
\m Certification authorities have the requirement to exchange cross certificates 
   with other certification authorities and make
   them publicly available.
\ei
Therefore the integration of public Directories (in particular X.500 Directories)
into the security infrastructure is a vital part of SecuDE.

A public Directory appears not only as information provider as part of the public security 
infrastructure, but also as user of security services in order to
protect its stored information from unauthorized access and to protect its 
communication. Security policies in case of the globally distributed X.500 Directory
must aim at three major goals:
\be
\m Protect the information base of the Directory.
\m Protect the internal and external communication of the Directory
   via DAP and DSP.
\m Protect the resources of the Directory.
\ee
These needs were the motivation for the development of the {\em Authentication
Framework} (X.509) as part of the X.500 Directory standard where {\em strong authentication} 
methods on the basis of digital signatures and certified key-to-name bindings
are being applied in order to protect the X.500 Directory. The certification procedures
and formats of X.509 are applicable in any context where public key cryptography is used. 
They play a central role in SecuDE. X.509 certificate
formats have been adopted, for instance, in the Internet PEM environment.

\subsection{Examples}
Following are a few examples where digital signatures on the basis of asymmetric
cryptography and confidentiality services on the basis of both symmetric and asymmetric
cryptography are essential:
\bi
\m {\em Electronic mail} is a very obvious example. The security of electronic
   mail must not be weaker than that of paper bound mail, i.e. it should 
   be possible to sign mail and to transmit it confidentially. 
   There are basicly two possible
   solutions to enhance electronic mail services with security features: 

   One is to add security information to message bodies without affecting
   the functionality and protocols of the message transfer systems involved.  
   The Internet RFCs 1421 - 1424, 'Privacy Enhancement for
   Electronic Mail' (PEM), provide such methods and
   coding formats for message encryption and authentication. Key certification is done here using 
   ASN.1-encoded X.509 cerificates. 

   The other solution is a full integration of security services into X.400 message handling systems which
   includes also the secure interworking of components of the message 
   transfer system.
   CCITT Recommendations X.400 (1988) / ISO 10021 (Motis) contain a comprehensive
   treatment of security issues including security specific protocol elements. 
   Key certification is again done by referring to X.509.
\m {\em Public directories} which store and provide public information of users to support
   electronic communication between them have two important security aspects:
   First, they have to provide
   security related public information like certificates and black lists.
   In this role Directories are part of the security infrastructure.
   Second, they have to control the access to the Directory Information
   Base (DIB). Only authorized
   persons should be able to modify directory entries, and read access 
   to directory
   data may be limited to certain persons. In this position Directories
   are users of security services.

   The CCITT Recommendations X.500 (1988) / ISO 9594 about directories 
   basicly provide two methods of protecting the DIB with the
   help of digital signatures: 
   \be
      \m {\em Strong Authentication} of the involved components during
            their association establishment (DUA-to-DSA and DSA-to-DSA).
      \m {\em Signed Directory Operations} when accessing the DIB.
   \ee
   The  {\em Authentication Framework} X.509 / ISO 9594-8 provides
   the basic procedures and formats for the application of public key
   based digital signatures. The certification
   techniques defined there are widely applicable outside the scope of
   directories, too.
\m For {\em document exchange} on the basis of ODA/ODIF security services are required, too.
   Similarly to electronic mail, it must be possible to sign or conceal whole
   documents or parts of them, so that sender and receiver of documents can trust
   the electronic exchange. ISO ODA/ODIF standards are currently going to be
   enhanced with features of that kind.
\m {\em Electronic data exchange} (EDI) is a common term for all kind of electronic
   data exchange in the commercial world (invoices, payment orders, trade data etc.).
   Security measures including digital signatures are necessary to protect EDI
   messages against both external attackers and inappropriate behaviour of the trading
   partners involved.
\m The correct operation of computer systems may be threatened by unauthorized 
   modifications (e.g. through viruses) of the soft- and hardware involved. Correct
   operation of the systems is the basis of all kind of application security.
   Beyond physical protection and well known techniques for the access to and maintainance of
   systems, digital
   signatures can be used to seal components of the systems through their lifetime. Those
   digital signatures can be verified when installing, loading or starting software
   and software programs.
\ei
Two aspects are common to all these examples concerning the introduction of security
services:

One is the need for a common authentication infrastructure which
provides key and certificate management. This authentication infrastructure must include
\bi
\m systems for asymmetric key and certificate generation, 
\m systems for asymmetric key and certificate distribution,
\m certification authorities (CAs) which are common points of
   trust for groups of users,
\m certificate revocation,
\m availability of public directory services which support the acces to 
   public user keys, certificates and certificate revocation lists. 
\ei

At the other end there is a need to store personal security related information items
like secret and public keys and certificates in a protected form in the 
local computer system.
\bi
\m Secret security-related information must remain secret, i.e. only the owner of the secret shall have
   read-access to it. Private keys can be considered as the digital
representation of individuals. Their release would allow masquerade
in various forms.
\m Public security-related information must remain unaltered, i.e. only the owner of this information
   shall be able to change it. An example of this kind of information is the last uncertified
   public verification key which the user has to trust in the end when he verifies a chain of 
   signatures and certifictes. This key may be read by everyone, but
   unauthorized modification of it would jeopardize the whole chain of trust.
\ei
 The protection of personal security related information can be supported by various technical means, for
instance through he use of smartcards. 
In SecuDE a protected area of the local system, which contains personal security related information, is called {\em Personal Security Environment} (PSE).

\section{What is SecuDE ?}
\markboth{What is SecuDE ?}{What is SecuDE ?}
\pagestyle{myheadings}
SecuDE (Security Development Environment) is a collection of utilities, APIs, 
and library routines which support the implementation of
application oriented security services and mechanisms in order to ensure authenticity and 
confidentiality of information objects on the basis of both symmetric and
asymmetric encryption methods and digital signatures. It is mainly intended to be
used by application programmers in order to add security functionality to applications.
Unix is the implementation platform of SecuDE. The authentification infrastructure
(key certification and distribution) is provided on the basis of X.509 defined 
procedures and formats and Internet PEM (RFC 1421 - 1424) defined procedures and formats.
\\[1em]
Particularly, SecuDE covers the following areas:
\be
\m A {\em security model} is provided which describes the basic 
   procedures and security services 
   to encounter given security threats (Volume 1, Principle of Security Operations).
\m {\em Basic mechanisms for security services} are provided. They include
   asymmetric and symmetric cryptographic functions in order to apply
   digital signatures and encryption to data items.
   A cryptosystem supporting RSA-, DSA- and DES-algorithms
   as well as hash functions (SqmodN, MD2, MD4, MD5, SHA), connected
   to a Personal Security Environment (PSE),
   is provided. The cryptosystem includes the RFC 1423 defined algorithm
   suite and DSS. The PSE is realized in two variants:
   \bi
   \m A smartcard environment. It consists of DES-capable smartcards
        and DES- and RSA-capable smardcard terminals with display
        and PIN pad. The smartcard terminals can be connected to the host
        computer via RS 232 interfaces.
   \m A software realization (called software-PSE) which is a 
         DES-encrypted Unix directory. The DES key is derived from
         a PIN which the user enters in order to gain access to the PSE.
   \ei
   A unique application interface hides the
   PSE technology differences (smartcard or DES protected software substitute)
   from the security applications.
 \m Functions and utilities for {\em X.509 authentication procedures}
   are provided.
\m Functions and utilities to support an {\em infrastructure of
   asymmetric key management}
   is provided.
   This allows distributed certification authorities to cooperate
   on certification of asymmetric public keys for users.
   Local functions enable users to generate, store and use
   their asymmetric keys.
\m Functions and utilities to access X.500 Directories are provided
   (integrated DUA). As a runtime option, X.500 Directories can be
   accessed automaticly by various SecuDE-functions, for instance in 
   order to retrieve certificates or certificate revocation lists. Another
   runtime option allows to access X.500 Directories using strong
   authentication and signed DAP operations.
\m Functions and utilities for the realization of PEM (RFC 1421 - 1424)
   are provided.
\m Utilities to sign, verify, encrypt and decrypt files and to maintain 
   own PSEs are provided.
\ee
All information items which are either stored locally or exchanged
within communication protocols are encoded according to ASN.1 Basic
Encoding Rules (ISO 8824/8825). 

In a separate distribution, QUIPU-8.0 with SecuDE based 
security-enhancements for strong authentication and signed DAP operations is available. This allows to operate secured X.500 DSAs.

\section{SecuDE Utilities Summary}
\markboth{SecuDE Utilities Summary}{SecuDE Utilities Summary}
\pagestyle{myheadings}


\subsection{User Utilities}

SecuDE provides a number of utilities to sign, verify, encrypt, decrypt 
and hash files, to perform PEM transformations, to maintain user's PSEs, to
operate certification authorities, and to perform key and certificate
management and distribution functions. All utilities require that
a user has a PSE which is his security database (a user may have
several PSEs representing different security contexts or linking
the user to different certification trees). Sectool(1) is the window-oriented
interactive tool to maintain a PSE, access X.500 Directories, and
exchange information between the Directory and the PSE.
Psemaint(1) is the line-oriented equivalent of sectool(1).   

\subsection{Utilities for Certification Authorities (CA)}
 
For the generation of certificates and CRLs certification authorities 
(CAs) are needed.
SecuDE basicly provides two methods for the exchange of security
information between SecuDE users and external CAs.
\be
\m A file interface of ASN.1 encoded objects is provided. These
   files can be exchanged by any means.
\m Procedures and formats according to PEM RFC 1424 are provided.
\ee
Key generation and certification can be done with
\be
\m User-generated keys, whereby the pubic component is
   subsequently transmitted to the CA for certification and
   retransmitted as certificate to the user (PEM
   procedures allow this method only),
\m CA-generated keys, whereby subsequently a whole PSE
   containing all keys and certificates is transmitted from
   the CA to the user (this requires that also the CA operates
   with SecuDE tools).
\ee
In addition, tools for setting up a CA hierarchy with SecuDE means 
only, and for the operation of CAs are provided.
This tool set is described in KM(1).

\subsection{Access to X.500 Directories}

SecuDE provides a set of functions to retrieve and update security 
attributes in X.500
Directories, like UserCertificate, CaCertificate, CrossCertificatePair, 
CertificateRevocationList etc. This DUA functionality is based on
QUIPU-8 software and requires a full ISODE-8 installation on the system. 
X.500 DUA functionality is integrated into many SecuDE utilities
and functions. For instance, sectool(1) and psemaint(1) have integrated
DUA functionality
which allows the transfer of security information between the user's
PSE and X.500 Directories and vice versa.

\section{SecuDE Library Functions Summary}
\markboth{SecuDE Library Functions Summary}{SecuDE Library Functions Summary}
\pagestyle{myheadings}

The SecuDE library consists of different functional groups or modules. The 
functionality of those modules is described by their programming 
interfaces.
 
The following modules are currently realized (Fig. 1):
\\ [0.4cm]

%\begin{figure}
\framebox[3.1cm][c]{\rule[-0.5cm]{0cm}{0.9cm}X.500}
\hspace*{0.39cm}
\framebox[3.1cm][c]{\rule[-0.5cm]{0cm}{0.9cm}PEM}
\hspace*{0.4cm}
\framebox[2.9cm][c]{\rule[-0.5cm]{0cm}{0.9cm}Key Managm.}
\hspace*{0.4cm}
\framebox[3.1cm][c]{\rule[-0.5cm]{0cm}{0.9cm}Other Applic.}
\vs{0.3cm}
{\bf
$|$ -- X500-IF -- -- $|$ \hspace*{0.2cm} $|$ -- PEM-IF -- -- $|$ \hspace*{0.3cm} $|$ -- KM-IF -- -- $|$
}
\vs{0.3cm}
\framebox[3.1cm][c]{\rule[-0.5cm]{0cm}{0.9cm}X.500 Support}
\hspace*{0.39cm}
\framebox[3.1cm][c]{\rule[-0.5cm]{0cm}{0.9cm}PEM Support}
\hspace*{0.4cm}
\framebox[2.9cm][c]{\rule[-0.5cm]{0cm}{0.9cm}KM Support}
\vs{0.3cm}
{\bf
$|$ -- -- -- -- -- -- -- -- -- -- -- -- -- -- -- -- -- AF-IF -- -- -- -- -- -- -- -- -- -- -- -- -- -- -- -- $|$
%$|$ -- -- -- -- -- -- -- -- -- -- -- -- AF-IF -- -- -- -- -- -- -- -- -- -- -- $|$
}
\vs{0.3cm}
\framebox[5.5cm][c]{\rule[-0.5cm]{0cm}{0.9cm}Local Certificate Handling}
\hspace*{0.4cm}
\framebox[4.10cm][c]{\rule[-0.5cm]{0cm}{0.9cm}Directory Access}
\hspace*{0.4cm}
\framebox[3.1cm][c]{\rule[-0.5cm]{0cm}{0.9cm}Auxil. Functions}
\vs{0.3cm}
{\bf
$|$ -- -- -- -- -- -- -- -- -- -- -- -- -- -- -- -- Secure-IF -- -- -- -- -- -- -- -- -- -- -- -- -- -- -- $|$
}
\vs{0.3cm}
\framebox[13.8cm][c]{\rule[-0.5cm]{0cm}{0.9cm}Personal Security Environment and Cryptography (technology dependant)}
\vs{0.3cm}
{\bf 
$|$ -- -- -- SCA-IF -- -- -- $|$
}
\vs{0.3cm}
\framebox[4.2cm][c]{\rule[-0.5cm]{0cm}{0.9cm}SC-Environment}
\hspace*{4.05cm}
\framebox[2.5cm][c]{\rule[-0.5cm]{0cm}{0.9cm}Crypto-SW}
\hspace*{0.2cm}
\framebox[2.5cm][c]{\rule[-0.5cm]{0cm}{0.9cm}SW-PSE}
%\vs{0.5cm}
%\caption{Interface Structure}
%\label{bild1}
\\[1em]
\stepcounter{Abb}

{\footnotesize Fig.\arabic{Abb}:
General SecuDE structure}
%\end{figure}

\subsection{Personal Security Environment and Cryptography (Secure)}
 
The cryptographic functionality and the PSE functionality
is strongly coupled in SecuDE. Keys which are being
used by the cryptographic functions are normally addressed
through references to PSE-objects. In the smartcard realization
the PSE is a separated device where cryptographic functions
are performed. The Secure module provides
the basic cryptographic functions for the generation and 
verification of digital signatures, and encryption and decryption 
of data, and the basic PSE access functions. No higher level
functionality like certificate processing is available in this module.
Two PSE realizations are availabe in this module, one of them
is a particular smartcard realization. Secure-IF, the interface of 
this module, however, hides different technologies from
the applications, i.e. Secure-IF claims to be an abstract local security
interface from the application's point of view which allows various
PSE realizations including smartcard environments without affecting 
the application's structure.
The description of the {\em Smartcard Application Interface} (SCA-IF) 
used in SecuDE is available on request.
Particularly, the Personal Security Environment and Cryptography 
module comprises
\bi
\m RSA, DSA and DES as basic cryptosystems (DSA only in software),
\m key generation,
\m different message digest algorithms (SqmodN, MD2, MD4, MD5, SHA), 
\m different signature and encryption algorithms (e.g. PKCS \#1),
\m routines to access the PSE, handling (create, delete, read, write)
   of PSE objects.
\ei

\subsection{Authentication Framework and Certification (AF)}

This module adds X.509 certification functionality to SecuDE.
Both local (i.e. PSE-located) certificates and directory-located
certificates can be addressed. Together with the SEC module, the
following functionality is provided:
\bi
\m sign, verify, encrypt, and decrypt information using PSE-based or 
   directory-based keys, certificates, certification pathes and 
   certificate revocation lists,
\m maintain a PSE with the objects defined in the AF 
   module, i.e. maintain certificates, crosscertificate pairs,
   the forward certification path, the public root key,
   certificate revocation lists, and lists of trusted public keys.
\m check PSE for validity and consistency,
\ei

\subsection{Key Management Support (KM)}
 
This module provides functions for the ge\-ne\-ra\-tion and distribution
of keys, certificates,
cross-certificates, and certification pathes, provides additional
functionality for the 
implementation of a Certification Authority (CA), and supports the
interaction between user and CA as well as between CA's.

\subsection{Privacy Enhanced Mail Support (PEM)}

This module covers functions 
which realize the Internet Specifications RFC 1421 - 1423. 
The basic idea of PEM is to define document oriented message 
encipherment and authentication procedures for the protection of 
messages through the use of 
end-to-end cryptography between originator and recipient with no special
processing
requirements imposed on the message transfer system (in contrast to
transfer 
protocol oriented security functionality as with X.400). This makes them
transparent to the mail 
transfer systems and applicable for local security services, too. 
Particularly, this module provides the following functions:
\bi
\m enhance messages, i.e. create MIC-CLEAR, MIC-ONLY and
   ENCRYPTED message formats,
\m de-enhance messages, i.e. scan MIC-CLEAR, MIC-ONLY and
   ENCRYPTED message formats, and perform the necessary
   validation steps (including CRL checks)
\m create Certification Request, Certification Reply, CRL Storage
   Request and CRL Storage Reply message formats,
\m scan Certification Request, Certification Reply, CRL Storage
   Request and CRL Storage Reply message formats, and perform
   appropriate actions.
\ei

\subsection{ASN.1 Encoding/Decoding Functions}

ASN.1 encoding and decoding functions are available 
for all SecuDE information objects which are intended for external
use (i.e. which are either stored externally, for instance in the
PSE, or exchanged via communication protocols). 
The encoding/decoding functions transform from the internal C-structures 
of the objects to corresponding (and in many cases standardized) ASN.1 codes
and vice-versa. The encoding functions generally apply distinguished encoding 
(DER) according to X.509 clause 8.7 in order to be able to process 
digitally signed objects.

\subsection{Auxiliary Functions}

A number of auxiliary functions are available for various purposes,
for instance
\bi
\m handling of distinguished names, algorithm identifiers, object
   identifiers and similar basic elements,
\m UTCTime handling functions,
\m error handling,
\m print routines
\m alias naming functions
\ei

\section{Next Version SecuDE-5.0}
SecuDE has continuing support at GMD. The next version SecuDE-5.0 will include, 
for instance:
\bi
\m Implementation of SecuDE in ANSI C.

\m An experimental implementation of GSS-API. 

   GSS-API is a generic security service application program interface
   developed in the Internet IETF. It is described at a (programming-) 
   language-independent conceptual level. In addition to this
   a C language binding exists. Both specifications are still drafts.
 
\m Support for the next generation smartcard. 

   SecuDE-4.1 supports the use of the GAO/GMD smartcard package Starcos 1.0 as
   one realization of the PSE. The Starcos terminal
   is a high-security crypto device which performs RSA and DES and provides a number of 
   physical protection features. It is therefore comparatively expensive. 

   Later versions of Starcos will have the Philips 83C852 chip on the smartcard as an 
   alternative. With this chip RSA (and possibly DSA) can be done on the smartcard, 
   and dumb and cheap smartcard readers can be used. It is anticipated that 
   SecuDE-5.0 will support this technology.
\ei

\section{Software and Documentation}

SecuDE is written in C and was developed on a Unix SunOS 4.1.2 platform.
Installations have been done on Sun/3 and Sun/4 systems running SunOS 4.0 - 4.1.3,
on HP 9000 system running HP-UX, on Siemens SINIX systems and on Apollo Domain systems.
SecuDE versions for MS-DOS and MacIntosh are in preparation.
SecuDE includes some third party public domain software, particularly ASN.1 software
from ISODE 8.0. Long integer arithmetic
needed for RSA and DSA is additionally available in a number of assembler packages for Sun/3, Sun/4,
HP 9000, Apollo 68000 based systems, and Intel 286/386 based systems, for efficiency reasons.

The {\bf documentation} related to SecuDe comprises three volumes.

{\em Volume 1} contains basic considerations
about principles of security operations and thus provides
background information about the specification of security interfaces.
It reveals
the authors' conception of secure and open communication
in that it describes
the components of security services
which support an open communication environment.
The style of Vol. 1 informal and
partly tutorial.

{\em Volume 2} contains the specification of security interfaces,
security library functions in terms of C-functions, security
utilities, and data structures in terms of C-structures
and ASN.1 structures.
The style of the specifications is in the form of 
Unix manual pages.
The reader is advised to refer to the background information
of volume 1 for tutorial explanations.

{\em Volume 3}  provides guidelines for the application of  SecuDE.
Applications contained in the SecuDE distribution are Privacy Enhanced 
Mail (RFC 1421 - 1423), Key Certification and Key Distribution, Security
Infrastructure Management, Secured X.500 Directory Access, and a 
SecuDE-based version of the Unix Xlock program where authentication
is done with digital signatures instead of password supply.

This list is open for further developments such as secure file transfer,
secure remote login, secure remote operations, etc.

The documentation is written
in \LaTeX. The corresponding DVI files and Postscript files are part of the distribution.

\section{References}
\markboth{References}{References}
\pagestyle{myheadings}

{\bf General}
{\small
\begin{description}
\item{[IsoSec]}
ISO 7489-2:
Information Processing Systems --
Open Systems Interconnection --
Basic Reference Model.
Part 2: Security Architecture, 1988.

\item{[ASN.1]}
ISO 8824:
Information Technology --
Open Systems Interconnection --
Specification of Abstract Syntax Notation One (ASN.1), 1992.

\item{[ASN.1BER]}
ISO 8825:
Information Technology --
Open Systems Interconnection --
Specification of Basic Encoding Rules for Abstract Syntax Notation One (ASN.1), 1992.

\item{[X500]}
CCITT: Blue Book Volume VIII -- Fascicle VIII.8,
Data Communication Networks:
Directory, Recommendations X.500-X.521 (1988).
Geneva, 1989.

\item{[X400]}
CCITT: Blue Book Volume VIII -- Fascicle VIII.7,
Data Communication Networks:
Message Handling Systems,
Recommendations X.400-X.420 (1988).
Geneva, 1989.

\item{[PEM93]}
Privacy Enhancement for Internet Electronic Mail.
IAB Privacy Task Force, January 1988 -- February 1993. \\
J. Linn:
RFC 1421, Part I: Message Encryption and Authentication Procedures.
February 1993. \\
S. Kent:
RFC 1422, Part II: Certificate-Based Key Management.
February 1993. \\
D. Balenson: 
RFC 1423, Part III: Algorithms, Modes, and Identifieres.
February 1993. \\
B. Kaliski:
RFC 1424, Part IV: Key Certification and Related Services.
February 1993.

\item{[PKCS1]}
PKCS \#1: RSA Encryption Standard, Version 1.4,
RSA Data Security, Inc., June 3, 1991.

\item{[TeleTrusT]}
TeleTrust: 1992, Object Identifiers -- Register.

\item{[ISODE7]}
M. Rose: The ISO Development Environment, User's Manual, Version 7.0, July 1991.

\item{[ISODE8]}
C. Robins, J. Onions: The ISO Development Environment, User's Manual Update, Version 8.0, June 1992.

\item{[GSSAPI]}
J. Linn: Generic Security Service Application Program Interface, Internet-Draft, April 1993
\end{description}
}
{\bf Algorithms}
{\small
\begin{description}
\item{[RSA]}
R. Rivest, A. Shamir, L. Adleman:
A Method  for Obtaining Digital Signatures
and Public Key Cryptosystems.
Communications of the ACM, 21(2), 120-126, Feb 1978.

\item{[ElGamal]}
T. ElGamal:  A  public key  cryptosystem  and  a signature
scheme  based  on discrete logarithms, IEEE  Transactions on
Information    Theory,  IT-31,  Number  4,  July  1985,  pp.
469-472.

\item{[NIST-DSA]}
Proposed FIPS XX, Digital Signature Standard, U.S. Dept. of Commerce /
National Institute of Standards and Technology (NIST), Feb. 1993 

\item{[NIST-SHA]}
Proposed FIPS YY, Secure Hash Standard, U.S. Dept. of Commerce /
National Institute of Standards and Technology (NIST), Oct. 1992 

\item{[OIW-Stable]}
OIW: Stable Implementation Agreements for Open Systems, 
Part 12 Security, March 1993

\item{[OIW-Working]}
OIW: Working Implementation Agreements for Open Systems, 
Part 12 Security, March 1993

\item{[DES]}
Federal Information  Processing Standards Publication  (FIPS
PUB)   46-1, Data  Encryption Standard,  U.S. Department  of
Commerce/National Bureau  of Standards, Supersedes  FIPS PUB
46,  January 15, 1977, Reaffirmed January 22, 1988. 

ANSI   X3.92-1981,  Data   Encryption  Algorithm,   American
National  Standards Institute, Approved December 30, 1980. 

Federal Information  Processing Standards  Publication (FIPS
PUB)    81,  DES  Modes  of Operation,  U.S.  Department  of
Commerce/National Bureau of Standards, December 2, 1980. 

ANSI  X3.106-1983,  Data  Encryption  Algorithm -  Modes  of
Operation, American  National Standards  Institute, Approved
May  16, 1983. 

Federal Information  Processing Standards  Publication (FIPS
PUB)  74, Guidelines for Implementing and Using the NBS Data
Encryption  Standard, U.S.  Department of  Commerce/National
Bureau of Standards, April 1, 1981.

\item{[MD2]}
B. Kaliski: RFC 1319,
The MD2 Message Digest Algorithm.
April 1992.

\item{[MD4]}
R. Rivest: RFC 1320,
The MD4 Message Digest Algorithm.
April 1992.

\item{[MD5]}
R. Rivest: RFC 1321,
The MD5 Message Digest Algorithm.
April 1992.
\end{description}
}
\end{document}
